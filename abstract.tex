\addtotoc{Abstract}  % Add the "Abstract" page entry to the Contents
\abstract{Past research related to exergames has found that they can help to motivate people to exercise by converting physical activity into an enjoyable game. However, these exergames have been single purpose usually home fitness only. In this thesis, we designed an exergame for warm up guidance and motivation to be used in gyms and fitness centers before physically more strenuous exercise. We utilized immersive technologies based on the hypothesis that they can be used as a guiding tool for warm up procedures, would increase warm up duration, and increase exercise enjoyment.  In order to evaluate our exergame we have conducted two user studies. In the first study we collected responses from 466 participants about their work out and warm up habits, as well as, general acceptance of our gamified solution of a warm up routine. In the second user study we comped a warm up procedure with a video showing a fitness instructor performing a warm up session to a warm up procedure with our exergame. In both conditions, the movements the participants  performed were identical. The usage of the exergame, showed a statistically significant increase in exercise duration relative to the non-gaming condition. 
Based on our findings from both the studies, we conclude that by making the exergame interactive and appealing, with intervals that last as long as the player chooses to, it is possible to transform the warm up procedure from a repetitive and tiresome activity to an entertaining and challenging necessity.TODO% The exergame showed a significant increase in user motivation and enjoyment when compared to the non gaming condition, with the Kinect condition found to be slightly more motivating than the screen condition.
\addtocontents{toc}{\vspace{1em}}  % Add a gap in the Contents, for aesthetics
%The current study employed an in-house developed exergame and manipulated the game features in a 2 (autonomy-supportive game features: on vs. off) × 2 (competence-supportive game features: on vs. off) experiment to predict need satisfaction, game enjoyment, motivation for future play, effort for gameplay, self-efficacy for exercise using the game, likelihood of game recommendation, and game rating.
\clearpage
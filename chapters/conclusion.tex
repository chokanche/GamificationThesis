\chapter{Conclusion}
\label{chap:conclusion}
In this paper we presented a development and evaluation process of an exergame for warm up guidance that is specifically targeted towards amateur athletes who rarely or never warm up before physically strenuous exercises and sports activities. We used Kinect sensor in order to track and capture players' movements and dispayed the exergame on the wall using a projector. By placing various game obstacle and coins in a specific position, our intention was to indirectly promote exercise through the gameplay of repeatedly performing warm up related movements chosen after related literature review and discussions with fitness experts.\\\\The development of our exergame consisted of three phases that included requirements gathering, prototype development with user evaluation, and final exergame development with further user evaluation. The first phase was an exploratory step in which we justified the development and identified currently available commercial and non-commercial solutions. In our research, we did not find any available solution with primary focus on warm up as a preparatory activity. In the second phase we implemented and evaluated our exergame prototype. The prototype was a scaled down version of our final exergame solution.  In order to evaluate our prototype, we created an online survey. The purpose of the survey was to explore general work out and warm up habits of the respondents, as well as their preferences and general acceptance of gamified solutions of warm up exercises.  Total number of n = 446 individuals participated in the online survey. With respect to respondents' warm up preferences before a physical activity, n = 251 (56.3\%) reported always warming up, whereas n = 195 (43.7\%) reported not warming up regularly before physically more demanding exercises. The results regading the reasons for avoiding warm up exercises aligned with the ones presented in \cite{fradkin2010effects}. This further justified the need for educational and motivational solutions, which are enjoyable and easy to carry out, with primary focus on the major muscle groups and benefits of warm up, in order to increase the proportion of athletes who engage in warm up routines before every exercise. The prototype exergame and one short warm up session has also been presented in the online survey. \\Total of n = 269 (60.31\%) respondents reported that would use the prototype exergame for warming up. Out of n = 195 respondents who reported not warming up before sports activities n = 124 (63.58\%) stated that they would use the presented solution as a tool for warming up before sports activities. Based on the results obtained from the survey, comments, and suggestions, in the third phase the prototype version of the exergame has been redesigned in order to better suit the needs of its future users. We opted for a modular design, consisting of multiple game segments each containing different obstacles that induced different movement during gameplay. These segments were generated procedurally on random and were easily modifiable. Our primary goal in the last phase was to investigate if our solution can be used to guide athletes through the warm up process efficiently. Despite some limitations, our exergame showed higher results and statistically significant difference in terms of exercise duration, physical activity enjoyment, and perceived exertion level compared to the non-gaming session under the same conditions. The exercise movements that were required to be executed during gameplay felt intuitive and came naturally to the participants. Thus, we concluded that the exergame provided adequate guidance in performing a general warm up procedure. Contrarily to expected results, the evaluation of psychological and emotional dimensions  did not show significant differences between two conditions. These results were most likely due to the fact that both conditions involved the usage of immersive technology and a novel approach (game with Kinect sensor and a warm up video) that succeeded in shifting the participants' focus from the discomfort and dullness of the exercise, but the results showed that the exergame condition offered a more immersive and enjoyable experience.\\\\
Based on the results obtained in this paper, we conclude that exergames with incorporated immersive technologies can be used  as a guiding tool for general warm up procedures and can offer significant benefits as motivational tools to promote engagement in warm up exercises for amateur athletes. Moreover, the exergame design given in this paper was shown to be effective in promoting the desired health outcomes in terms of increased range of motion and expected heart rate recommended by previous studies and medical experts.  Lastly, the modular design approach that have been followed in the development of the final version of our exergame solution has been effective for game segment customizations and adjustments. This, further, makes the game easily extensible in case additional movements need to be incorporated for future use cases. Future studies investigating exergames should consider the design approach introduced in this paper, such as the modular design of the game segments, the procedural generation of the exergame environment, and the advantages of immersive technologies.\pagebreak
\section{Future Work}
This paper has tackled several interesting areas worth considering in future works. 
Firstly, our solution has been designed for general warm up exercises in mind. This means that the movements induced in various game segments could be executed without any prior knowledge of the exercise. However, based on the surveys conducted and the results collected, the respondents who declared themselves as professional athletes found the required movements too general and not applicable for the sports they engage in. Hence, generating versions of the exergame that are designed for certain sports areas, and require previous knowledge of the movements might be worth considering. This way, our solution could be used not only in gyms by amateur athletes, but have more diverse players pool. Secondly, due to hardware constraints, our solution did not take into account the correctness of the movements executed. As pointed out, we opted for general movements that are intuitive and easily executed. Interestingly, we observed that some of the players struggled at the beginning of the gameplay to execute the movements correctly. This resulted in two scenarios. Either the player missed collecting the coin and hit the obstacle, or the coin was collected but the movement was faulty. It would be interesting to see how effective our solution would be in not only guiding the players through a warm up procedure, but in the same time, correcting the movements that are not executed properly. Lastly, survey respondents and study participants indicated an interest of having a possibility for a collaboration during gameplay in a form of a group warm up or a competition. Previous studies already showed that introducing competition in a form of scoreboards can be a powerful motivator for certain player types \cite{ werbach2012win, zichermann2011gamification}. Having this in mind, our solution also incorporated game elements suitable for more competitive player types and assessed their effects and acceptance. Even though our exergame in it's current design is not well suited for collaborative warm up session that include multiple players at the same time, the existing high score system could be extended in a way the results are shared on social media platforms further enhancing the competitive aspect. Alternatively, the exergame could be extended in a way it allows multiple players to perform a warm up procedure in the same time, either collaborating or competing among each other. 
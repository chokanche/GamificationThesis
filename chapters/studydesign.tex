\chapter{Study Design}\label{chapter:studydesign}

\section{Description of the Experiment}

In order to critically evaluate our system, after the features of our prototype exergame are evaluated in the first stage and the second version implemented, we pilot-test and assess it through a second survey. In the this stage, we test the motivational and guidance features of our system. For this purpose a between subject experiment is conducted with 20 participants. 

\subsection{Introduction and Goals} 
Introduce your experiment, and give the reader the specific goals you expect it to address. It is common at this stage to give the reader a hint of your hypotheses (if they are not already hinted at in the Introduction).
\subsection{Methods} 
In order to address the research questions outlined in XX, we implement and test an exergame for warm up exercises before sports activities. 
%This is a detailed description of the experiment that should allow other researchers (familiar with HCI and experimental design in general, but not familiar with your experiment) to replicate your experiment.%
\subsubsection{Participants}
Describe your participants (e.g., any relevant demographics, if/how they were divided into categories), including total number, and recruiting approach. Indicate if any incentives were used. Comment on the representativeness of your participants relative to the target population, if their representativeness isn’t immediately obvious.
\subsubsection{Conditions}
If your experiment is comparing multiple different interfaces or interactive systems or techniques, describe each of them. Screen snapshots of interfaces/systems are particularly useful.
\subsubsection{Tasks}
Briefly describe what participants were asked to do with the interactive system(s).

\subsubsection{Design}
Write the formal experimental design (e.g., a 2 x 3 mixed factorial design, more specifically a 2 levels of expertise (between subjects) x 3 interfaces (within subjects) design).
\subsubsection{Procedure}
%Describe the sequence of activities each participant followed. This should document the experiment from a participant’s perspective, from the moment s/he arrives (e.g., a preliminary questionnaire to obtain X information, followed by five tasks with system A, then a 10 min break, followed by the same five tasks with system B, and finally a semi-structured interview to solicit opinion on Y).%
The activities each participant of the experiment session followed is presented in this section.\\ 

Before the experiment, the lab environment is set up. The Kinect sensor is placed in a correct position, the projector is turned on. In each session only one participant is present and guided by the researcher. The activities each participant >>> are as follows:
\begin{itemize}
\item The researcher explains the sensors and tools that are required for the experiment, after which the participant puts them on. The sensors used in each session include a heart rate monitor and Microsoft Band. In order to measure the range of motion around a joint in the body, a goniometer is utilized. %maybe kinect could do it%
After the researcher confirms that the sensors are placed in a correct position, we start recording heart rate data and WHATEVER WE DO WITH THE BAND.
\item For each participants we measure ROM of the following joints using double-armed goniometer. %ovo mozda na pocetku%
\item After the measurements are completed, the participant rests for up to 10 minutes in order to take the readings of the resting heart rate. 
\item While the participant rests, the researcher explains and presents the movements that are required from the participant to perform during the experiment. Furthermore, the researcher explains how the game is played.
\item When the rest period completes, the participant is asked to practice the required movements.
\item In order to avoid starting the game and warm up with already stimulated heart rate, the participant is required to rest for 5 minutes. 

\item The participant is asked to prepare for the warm up by positioning to the position marked by the researcher. 
\begin{itemize}
\item  If this participant is part of the experimental group, the game starts with the start scene where the participant enters his or her name. After 5 seconds, the game proceeds with scenes in which the participant performs the previously presented movements in order to avoid obstacles and collect coins. The duration of the game is not fixed and it is played up to the point when the participant feels warmed up enough. During the experiment, the warm up procedure performed by the participant is recorded. %this reformulate%

\item  If this participant is part of the control group, the video that shows a gameplay performed by another participant who was part of the experiment group. The participants performs the same movement as in the playing video. As with with the sessions in the experiment groups, the duration of the warm up is not fixed and the 
video is played up to the point when the participant feels warmed up enough. During the experiment, the warm up procedure performed by the participant is recorded.
\end{itemize}
\item After the participant finished with the warm up, he or she takes a rest. During this period the researcher assesses the ROM of the participant. 
\item The participant in the experiment group plays the game and the participant in the control group watches the video for the second time with the same content as previously. This results in more heart rate data, and allows more opportunity for the player
to experience psychological flow, as discussed in section 3.6.1. Another 10 minutes of play time
brings the total exposure to about 20 minutes plus the break. OVO MODIFY
\item With the game complete, the participant removes the sensors, rests and completes the survey. MODIFY
\end{itemize}
\subsubsection{Apparatus}
Describe the physical setup of the experiment (e.g., where was it conducted, on what kind of equipment, etc.)
\subsubsection{Independent and Dependent Variables}
Include exactly how you intend to measure each dependent variable. 
\subsubsection{Hypotheses}
Remember to state these in terms of the independent and dependent variables. If it is not immediately clear why you would have a certain hypothesis (it often follows logically from the introduction of the experiment), then include a brief explanation separate from but following the hypothesis. You do not need to state the null hypothesis.
\subsection{Problems/Limitations}
Describe any problems/limitations encountered that will help other researchers avoid or account for them if they decide to replicate your experiment.
\section{Results}

This section is an objective report on what the numbers show. You should not try to interpret the meaning of the numbers in this section. Some of the things you may do here are: 
report means and standard deviations in neat tables 
indicate the statistics used and levels of significance 
include graphs, plots, histograms, etc that tell a story about the actual figures obtained 
Only critical raw data and summary statistics should be included in the actual report. However, you must keep all your raw data in a separate archival report, should anyone (a reviewer in the case of real scientific reporting) need more detail than is provided in the paper. 
\section{Discussion}
Interpret the results. Although you should still try to be as objective as possible, the discussion section should illuminate your critical thinking about the results. Explain what the statistics mean, account for anomalies, and so on.
\subsection{Interpretation of Results}
Discuss what you believe the results really mean. For example, if you find a significant difference for some effect, what does that mean to the hypothesis? Is the different seen an important one?
\subsection{Relation to other works}
How do the results you’ve obtained relate to other research findings?
\subsection{Impact for practitioners}
As computer scientists, we are particularly concerned with the implications of our findings on practitioners. Should existing interface constructs be designed differently or used in a new context? Do you have suggestions for new designs? How can the findings be generalized?
\subsection{Critical reflection}
Critical reflection is one of the key foundations of science. You should criticize your work (constructively, if possible), indicate possible flaws, mitigating circumstances, the limits to generalization, conditions under which you would expect your findings to be reversed, and so on.
\subsection{Research agenda}
The best experiments suggest new avenues of exploration. In this section, you should reflect and refine your hypotheses, describe new hypotheses, and suggest future research, ie research that you would do if you continued along this path.
\section{Conclusions}
Summarize the report, and speculate on what is to come.
Acknowledgements. This section should give thanks to the major people (supervisors, associates) and organizations (sponsoring agencies, funders) that helped you. For example, I would like to thank Ben Shneiderman, whose report framework was used to build this one.
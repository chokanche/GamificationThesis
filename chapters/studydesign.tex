\chapter{Study Design}\label{chapter:studydesign}
The main goals of this research were to develop an exergame which can be used for warming up exercise before more strenuous physical activity and to evaluate its effectiveness. For this purpose, (to specify which framework) has been utilized.  In this chapter we outline the research framework used and detail and discuss the research methods. 
\section{Description of the Experiment}
This section describes the evaluation of the second version of the Immotion exergame. For this purpose, an approach was adopted that uses a mixture of different tools and user study methods. This period of testing of our gamified solution is known as pilot testing. During this period, data has been logged, surveys have been conducted and interviews undertaken. Similar to first evaluation, the obtained results are analyzed in order to determine to which level our proposed solution was effective in the context and whether it offered a solution to the problem. Based on the results from the exergame prototype evaluation (Chapter 1) and a variety of flow research presented (Chapter 2), guidelines were followed which influenced the design and development of the second version of the Immotion exergame. 

\subsection{Introduction and Goals} \label{chapter:goals}
The primary goal of the second study was to investigate whether our exergame solution can be used as an interactive guide for individuals who do not know how to perform warm up routines; can it be used as a solution that motivates individuals to warm up regularly before physically more demanding exercises, encourages behavior change, and provides an enjoyable exergame experience. \\\\\\Taking this into account, the aims of this evaluation included: 
\begin{enumerate}
\item Comparing the warm up guidance experience between participants that use the exergame and participants that do not use the exergame.
%\item Comparing the experience between participants that use the exergame and participants that do not use the exergame. 
\item Analyzing the effect on the motivation to undertake warm up procedure, and behavior change, when using the exergame compared to participants who are not using the exergame.
\item Identifying if the exergame was enjoyable and easy to use.
\end{enumerate}
In order to evaluate the effectiveness of our gamified solution in the given context, the user base is divided into two groups: \textit{experimental group} and \textit{control group}. The first, experimental group, is the one that interacts with the exergame directly. Contrarily, the control group is presented with the video of a participant who interacted with the system, and does not engage with the exergame directly. This approach allows us to infer the influence of our gamified solution, as well as, to assess the main differences in completing the required activities between the two user groups. 
\subsubsection{Assumptions}
\begin{itemize}
\item The participants will be able to perform the requested movements.
\item The participants will be able to play the exergame for at least 1 minute.
\item The participants will answer all the questionnaires truthfully.
\item The software and hardware that is used used will function properly.
\end{itemize}
\subsubsection{Hypotheses}
%Remember to state these in terms of the independent and dependent variables. If it is not immediately clear why you would have a certain hypothesis (it often follows logically from the introduction of the experiment), then include a brief explanation separate from but following the hypothesis. You do not need to state the null hypothesis.%
Based on the aims of the study outlined in Section  \ref{chapter:goals} the following hypotheses were established that could be tested: 
\begin{enumerate}
\item Participants that are interactively guided through the warm up procedure using the exergame are positive experience compared to the participants have to warm up without the exergame.
\item Participants had a more positive warm up experience when using the exergame compared to the participants not using the exergame.  
\item Participants felt more motivated to undertake warm up exercises when using the 
exergame compared to the participants not using the exergame.
\item The exergame will encourage more 
behavior change compared to the standard warm up procedure. 
\end{enumerate}

%Introduce your experiment, and give the reader the specific goals you expect it to address. It is common at this stage to give the reader a hint of your hypotheses (if they are not already hinted at in the Introduction).%
\subsection{Methods} 
In this section we outline the methodology adopted for the Immotion exergame evaluation.
We decide to follow a mixed-methods approach, and by doing so, utilize both qualitative and quantitative data sources in combination.
%
%This is a detailed description of the experiment that should allow other researchers (familiar with HCI and experimental design in general, but not familiar with your experiment) to replicate your experiment.%
\subsubsection{Participants}
Total of n = X individuals participated in the study that has been conducted DATE in DFKI. All participants were students from Saarland University. For recruiting participants, posters were distributed in print, and sent through social media and email (Appendix X). Each participant was given X euros for taking part in the study. All of the participants were amateur athletes who engage in some physical activity few times per week. For the study we particularly targeted individuals who exercise in gym or fitness centers and often avoid preforming warm up exercises before more strenuous physical activity. All participants were required to report to the laboratory in gym based
clothing, preferably shorts and t-shirt, and all of them performed the required tests in the same location using the same equipment. Before the study, each participant signed a consent form (Appendix X). TODO: This should be updated later with real data.
%Describe your participants (e.g., any relevant demographics, if/how they were divided into categories), including total number, and recruiting approach. Indicate if any incentives were used. Comment on the representativeness of your participants relative to the target population, if their representativeness isn’t immediately obvious.%
\subsubsection{Conditions}
Each participant of the study took part in a single test session one hour in duration. During this session, all the participants completed a pre-test questionnaire (Appendix X), after which they performed two exercise sessions, separated by a 10 minutes break. At the end of the session the participants completed a post-test questionnaire (Appendix X). Two conditions were evaluated:
\begin{enumerate}
\item Exercising with the game projected on a wall in front of the participant.
\item Exercising without the game with a video of a participant playing the exergame projected on the wall in front of the participant.
\end{enumerate}
TODO: third condition participants who warm up without video or the exergame?\\
Depending on the group, each participant performed exercise that represent one of the conditions. The participants are assigned to each group randomly.
At the beginning and the end of each exercise session, the measurements for participants' ROM are taken. For this purpose, a goniometer is utilized. Additionally, participants' heart rate were measured and recorded using XX. TODO: say the purpose of Microsoft Band. 
%If your experiment is comparing multiple different interfaces or interactive systems or techniques, describe each of them. Screen snapshots of interfaces/systems are particularly useful.%
\subsubsection{Tasks}
In order to interact with the gamified system, the participants in the experimental group were required to perform a set of general movements. By performing these movements, the participant controls the game avatar and, by doing so, avoids obstacles and collects coins. Based on the data collected from the first survey, in order to successfully finish the game, only movements that are, first, detectable with high accuracy using only one Kinect device and, second, simplistic enough to be accomplished easily without no prior exercise knowledge or experience were required. The movements the participant needed to perform included: 
\begin{itemize}
\item right hand movement up,
\item left hand movement up,
\item jump right,
\item jump left,
\item jump up, 
\item star jump, and
\item squat.
\end{itemize}
Participants who were in the control group and did not interact with the gamified system were required to perform the same set of movements. Participants in this group had to follow a video that was projected on the wall in front of them. The video was a recording of a participant from the experimental group who interacted with the gamified system. By following the video, the participant was required to execute the same movements as the participant in the presented video.
%Briefly describe what participants were asked to do with the interactive system(s).%

\subsubsection{Design}
Write the formal experimental design (e.g., a 2 x 3 mixed factorial design, more specifically a 2 levels of expertise (between subjects) x 3 interfaces (within subjects) design).
\subsubsection{Procedure}
%Describe the sequence of activities each participant followed. This should document the experiment from a participant’s perspective, from the moment s/he arrives (e.g., a preliminary questionnaire to obtain X information, followed by five tasks with system A, then a 10 min break, followed by the same five tasks with system B, and finally a semi-structured interview to solicit opinion on Y).%
%The activities each participant of the experiment session followed is presented in this section.\\ 
Before the experiment, the lab environment is set up. The Kinect sensor is placed in a correct position and turned on. The PC running the software is started and the projector is enabled. In each session only one participant is present and guided by the researcher. The activities each participant followed are:
\begin{itemize}
\item The participant completes the pre-test survey.
\item The researcher explains the sensors and tools that are required for the experiment, after which the participant puts them on. The sensors used in each session include a heart rate monitor and Microsoft Band. In order to measure the range of motion around a joint in the body, a goniometer is utilized. TODO: use kinect for this?%maybe kinect could do it%
After the researcher confirms that the sensors are placed in a correct position, we start recording heart rate data. %what do we do with the band?
\item For each participants the researcher measures the ROM of the following joints using double-armed goniometer: to be discussed. %ovo mozda na pocetku%
\item After the measurements are completed, the participant rests for up to 10 minutes in order to take the readings of the resting heart rate. 
\item While the participant rests, the researcher explains and presents the movements that are required from the participant to perform during the experiment.
\item When the rest period completes, the participant is asked to practice the required movements.
\item In order to avoid starting the game and warm up with already stimulated heart rate, the participant is required to rest for 5 minutes. 
\item The participant is asked to prepare for the warm up by positioning to the spot marked by the researcher.
\item The researcher starts recording the session using to be discussed.
\begin{itemize}
\item  If this participant is part of the experimental group, the game starts with the start scene where the participant enters his or her name. After 5 seconds, the game proceeds with scenes in which the participant performs the previously presented movements in order to avoid obstacles and collect coins. The duration of the game is not fixed and it is played up to the point when the participant feels warmed up enough. During the experiment, the warm up procedure performed by the participant is recorded. %this reformulate%
\item  In case the participant is part of the control group, the video that displays a gameplay performed by another participant who was part of the experiment group is presented instead of the exergame. The participants performs the same movement as in the playing video. As with with the sessions in the experiment groups, the duration of the warm up is not fixed and the 
video is played up to the point when the participant feels warmed up enough. During the experiment, the warm up procedure performed by the participant is recorded.
\end{itemize}
\item After the participant finished with the warm up, he or she takes a rest. During this period the researcher assesses the ROM of the participant. 
\item The participant in the experiment group plays the game and the participant in the control group watches the video for the second time with the same content as previously.% This results in more heart rate data, and presumably allows more opportunity for the player to experience psychological flow, as discussed in section 3.6.1. 
\item After the participant finished with the gameplay (or video) for the second time, the sensors are removed.
\item The participant rests and completes the post-test survey. 
\end{itemize}
\subsubsection{Apparatus}
Describe the physical setup of the experiment (e.g., where was it conducted, on what kind of equipment, etc.)
\subsubsection{Independent and Dependent Variables}
Include exactly how you intend to measure each dependent variable. 

\subsection{Problems/Limitations}
Describe any problems/limitations encountered that will help other researchers avoid or account for them if they decide to replicate your experiment.
\section{Results}

This section is an objective report on what the numbers show. You should not try to interpret the meaning of the numbers in this section. Some of the things you may do here are: 
report means and standard deviations in neat tables 
indicate the statistics used and levels of significance 
include graphs, plots, histograms, etc that tell a story about the actual figures obtained 
Only critical raw data and summary statistics should be included in the actual report. However, you must keep all your raw data in a separate archival report, should anyone (a reviewer in the case of real scientific reporting) need more detail than is provided in the paper. 
\section{Discussion}
Interpret the results. Although you should still try to be as objective as possible, the discussion section should illuminate your critical thinking about the results. Explain what the statistics mean, account for anomalies, and so on.
\subsection{Interpretation of Results}
Discuss what you believe the results really mean. For example, if you find a significant difference for some effect, what does that mean to the hypothesis? Is the different seen an important one?
\subsection{Relation to other works}
How do the results you’ve obtained relate to other research findings?
\subsection{Impact for practitioners}
As computer scientists, we are particularly concerned with the implications of our findings on practitioners. Should existing interface constructs be designed differently or used in a new context? Do you have suggestions for new designs? How can the findings be generalized?
\subsection{Critical reflection}
Critical reflection is one of the key foundations of science. You should criticize your work (constructively, if possible), indicate possible flaws, mitigating circumstances, the limits to generalization, conditions under which you would expect your findings to be reversed, and so on.
\subsection{Research agenda}
The best experiments suggest new avenues of exploration. In this section, you should reflect and refine your hypotheses, describe new hypotheses, and suggest future research, ie research that you would do if you continued along this path.
\section{Conclusions}
Summarize the report, and speculate on what is to come.
Acknowledgements. This section should give thanks to the major people (supervisors, associates) and organizations (sponsoring agencies, funders) that helped you. For example, I would like to thank Ben Shneiderman, whose report framework was used to build this one.
\chapter{Behaviorism}\label{chapter:behaviorism}

The player forms the root of Gamification and, in any system, the outcome is affected and driven by his motivation (Zichermann \& Cunningham, 2011, p. 15). Therefore, to understand the potentials and fundamental aspects behind Gamification, one important part is to understand what drives people's motivation. Thus, psychology is  essential  to  Gamification  in order to understand  how  human  nature  works  and  how  it can  be  influenced  in  order  to  create  an  effective  gamified  system. For this reason, the next sections introduce different views from psychology about motivation and explain what has to be considered in terms of truly engaging individuals.  

There are three main purposes of this section. The first 
part is to provide a good overview of the subject itself and to introduce terms that will be used later in the discussion. The  second  purpose  is  to  present  theories  that  show  the engaging  and  motivating  effect of games
,  while the  third  is  to  establish  the  frame  that  constitutes  a  game  that  was  used  when 
designing the 
Gamification system.
Chapter 4 will be built upon this foundation
, attempting  to 
extend it by
presenting some ground rules based 
upon the 
empirical findings
and incorporating 
the perspectives and concerns of change managers and 
Gamification experts
\subsection{The Rules of Motivation}

The word \textit{motivation} originates from Latin \textit{motivus} and stands for ``serve to move''. In other words, motivation can be interpreted as \textit{to be moved to do something}. It can be defined as ``those forces within an individual that push or propel him to satisfy basic needs or wants'' \cite{pardee1990motivation}.  % A motive is what prompts the person to act in a certain way, or at least develop an inclination for specific behavior. 
According to Zichermann \& Cunningham, there exist four underlying reasons why people are motivated to play games, which can be viewed together or separately as individual motivators (Zichermann \& Cunningham, 2011, p. 20). These reasons are as follows:
\begin{itemize}
\item For mastery
\item To destress
\item To have fun
\item To socialize
\end{itemize} 

People play games not so much for the game itself as for the experience
that the game creates: an exciting adrenaline rush, a vicarious adventure, a mental challenge;
and the structure games provide for time, such as a moment of solitude or the company of
friends.
Nicole Lazzaro, an expert on player experience and emotions in games, 

%%TODO: OVO MOZDA NA KRAJU....
proposes three innate psychological needs to be crucial for optimal human development, functioning and well-being: competence, autonomy, and relatedness.%extend??
SDT
\section{SDT}

Another aspect to understanding player motivations is by questioning the source of one's motivation. One of the most influential motivational theories is the Self Determination Theory (SDT) introduced by Ryan \& Deci. It is an empirically derived theory of human motivation that makes distinctions between different types of motivation in terms of reasons and goals that cause the respective action. In general, one can distinguish between intrinsic and extrinsic motivation. The first type of motivation, as the word \textit{intrinsic} already suggests, refers to doing an activity for the inherent satisfaction and sense of drive that emerges from within. When intrinsically
motivated a person is moved to act because the activity is challenging, interesting and enjoyable on its own rather than because of external prods, pressures, or rewards. On the other hand, extrinsic motivation refers to performing an action because it leads to \textit{separable outcome}. That is, there is some external reward or influence which drives the person to accomplish the task (Deci  \& Ryan,  2000). 

According to Ryan \& Deci (2010), each person has different amounts and also different kinds of motivation. That is, each person is different in level (i.e. how much motivation) and orientation (i.e. what type of motivation) of their motivation, whereas orientation might be a goal which give rise to action. 

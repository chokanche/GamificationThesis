\chapter{Introduction}
\label{chap:intro}
Physical activity and exercise can have immediate and long-term health benefits. It significantly decreases the commonness of chronic diseases, serves as a countermeasure for psychological disorders, and greatly limits the severity of episodes of anxiety and depression \cite{mayr2015prevention, warburton2006health}. The counterpart to all these benefits is that engaging in physical activity is often associated with a higher risk of injury which can occur in athletes of all age and types \cite{van1997severity}. However, most of the injury types can be prevented. There exist different injury prevention mechanisms that are suggested by sports professionals and physical therapists. They all agree that every physical activity must begin with a warm up procedure.  This moderate activity prepares the athlete's body for the more intense exercise, improves the subsequent performance, and can decrease the likelihood of injury. \\\\Despite the benefits of engaging in warm up, it is still avoided by many recreational and professional athletes. Reasons for doing so are various. However, it boils down that it is a boring and time consuming activity. That is, athletes are not motivated sufficiently to engage in this activity on a regular basis. This suggests that educational and motivational solutions with primary focus on the benefits of warm up, including injury prevention, need to be developed and implemented in order to increase the proportion of athletes who engage in warm up routines before every strenuous exercise.\\\\  Lack of motivation may cause an athlete not to engage in warm up exercise, or not to engage at a proper intensity or for a sufficient duration. One approach to solve the motivation problem in athletes is to combine the warm up exercise and entertainment in the form of physical exercise video games (exergames) by utilizing game elements that can increase the motivation and engagement (gamification). Currently, there exists multiple successful commercial exergaming products \cite{wii, dance}. Also, the amount of research and publications on this topic have  been increasing rapidly in recent years.\\ However, all these exergames are intended for home usage and general fitness. There is an absence of exergames that are specifically designed and developed to be used for warm up exercises and targeted towards individuals who avoid warming up before sports activities. Furthermore, only few studies investigated the effects of immersive technologies on exergaming and individuals motivation to engage in physical activity []. These studies have shown that immersive technologies in the fitness domain have a potential to improve participant motivation because they can immerse the exergame players sufficiently so that their focus is shifted from the discomfort and exertion of the exercise towards the enjoyment of the experience. Nevertheless, as in other studies, these exergames were designed for general fitness routines, and often lacked meaningful gameplay that fits the exercise being performed.  Thus, utilizing these technologies with gamification elements opens up some exciting possibilities for research on increasing motivation in athletes to  engage in warm up exercises before physically more demanding activities. 

\chapter{Literature review}\label{chapter:warmup}
%Deterding et al. go on to explain that a gamified system, like a game, usually has rules and is goal oriented. Gamification is not a full-fledged game, although it utilizes many techniques that games implement such as, levels, clear goals, time constraints, badges, value conscious game design, challenge, limited resources and leader boards. It allows people to stay grounded in reality, whilestill profiting from successful game benefits like gaining access to a person’s emotions and intrinsic motivation, which in turn will help create habits. ZA ABRSTACT??
The following chapter provides an overview of the past research related to the warm-up as a preparatory exercise prior performing physical activity and conceptually related works in the domain of gamified solutions relevant to fitness and exercise. To impose structure, first the basic concepts related to warm-up and an overview of studies and results regarding the benefits of warm-up is given. Following, the concept of gamification is introduced. At the end of this chapter, a comparison between the presented approaches and our solution is made.
\section{Warm up in sports}
\subsection{Introduction}
Despite very contrasting believes and limited scientific evidence regarding it's effectiveness in many situations, warm-up (WU) has become a standard practice among professional and
recreational athletes \cite{bishop2003warm1, bishop2003warm2, shellock1985warming}. WU in sports is defined as a period of preparatory exercise which is carried out in order to prepare the athlete for the demands of the subsequent physical activity \cite{karvonen1992importance, woods2007warm, hedrick1992exercise}.
%karvonen skini i procitaj
Typically, WU includes a short and low-intensity preparatory activity which is followed by a stretching routine and sport specific exercise \cite{safran1989warm}. 
The purpose of WU is to enhance the subsequent competition or training performance and improve muscle dynamics to reduce the risk of sport-related injury\cite{bishop2003warm1, shellock1985warming, knudson2008warm}. 
 %HERE MENTION THE STUDIES THAT SAY THIS DOESNT HELP.  CHECK LEMON ILIEV
Nonetheless, there is still deficiency of scientific evidence on what kind of WU can influence both muscle damage prevention and performance improvement \cite{safran1989warm}.\\ %cite
\subsection{WU benefits}
Fradkin \textit{et al.} carried out a systematic review and meta-analysis of relevant studies concerning the benefits of WU on the performance. They found that an adequate WU supports an improvement in performance in 79\% of the research studies analyzed. Furthermore, they pointed out that there exist little evidence supporting detrimental effects WU might have on performance and sports participants.
WU can affect the performance via variety of temperature and non-temperature related mechanisms \cite{bishop2003warm1}. 
%By performing a low intensity training routine before taking part in more demanding exercise, an increase in one's body temperature \(Magnusson 
%et al., 2000\) and muscle blood flow occurs \(Tiidus \& Shoemaker, 1995\).  
The most relevant effects of WU can be attributed to physiological mechanisms like increased muscle temperature, decreased resistance of muscle and joints (decreased stiffness), increased oxygen delivery to muscles, increased nerve-conduction rate and speeding of metabolic reactions \cite{bishop2003warm1}. 
However, the benefits of WU are not exclusively physical. Apart from the physiological changes a body undergoes during this preparatory period, it has been hypothesized that a possible psychological benefit can also be gained by following a proper WU routine \cite{bishop2003warm1,shellock1985warming}.
It has been suggested that WU can serve as preparatory phase providing time for athlete to concentrate and mentally prepare for the forthcoming exercise \cite{shellock1985warming}. 
%Thus, possible psychological benefits is increased mental preparedness for the forthcoming exercise\cite{bishop2003warm1}. 
For instance, in the study that investigated the link between a WU and psychological processes, Ladvig (2013) reported that athletes who reported performed a proper WU routine before engaging in more demanding physical activity demonstrated significantly higher levels of exercise related motivation and enjoyment. Thus increased motivation and enjoyment is an additional psychological benefit of WU \cite{ladwig2013psychological}.
 %findings of this theisis: http://aut.researchgateway.ac.nz/bitstream/handle/10292/325/WeerapongP.pdf?sequence=1
\\Apart from physiological and psychological benefits, WU has been suggested to have an important role in sports-related injury prevention \cite{shellock1985warming}. Unfortunatelly, there exist no high-quality research studies in order to draw definite conclusion as to the effect WU has on sports-related injury prevention \cite{fields2007should}. Safran, \textit{et al.} proposed a possible bio-mechanical explanation for injury reduction with WU. The results of this study reported that warmed-up muscles in the animal models can elongate more before failure caused by increased force and length of stretch \cite{safran1989warm}. In the study of Fradkin, \textit{et al.} the current evidence regarding WU in injury prevention has been assessed. Out of five high-quality studies with sufficient data that have been systematically reviewed, three studies reported significant injury related reduction by performing WU before the physical activity while in other two no benefits were reported. Overall, based on the weight of evidence, it has been concluded in favor of WU to decrease the risk of injury.\\
%Furthermore, Nosaka and Clarckson found that high and low intensities of WU could reduce the magnitude musculatory damage ... They proposed that 
%A search of the literature identified only one published research paper on the effects of warm-up on the severity of muscle damage (Nosaka & Clarkson, 1997).  Nosaka and Clarkson (1997) found that both high (100 repetitions of maximal concentric contraction) and low (100 repetitions of minimal concentric contraction) intensities of 
%warm-up could reduce the magnitude of mu
%scle damage as indicated by reduced 
%soreness sensation, strength and range of 
%motion loss, swelling, and creatine kinase 
%activity.  The authors proposed that warm-up 
%might help to increase muscle temperature 
%and circulation, and consequen
%tly, increase muscle and conne
%ctive tissue el
%asticity
%the majority of effects of warm up have been attributed to temperature related mechanism
\subsection{WU types}
There exist various types of WU procedures professional and recreational athletes use as a preparatory phase for the physically more demanding exercise preparation. It is important to distinguish between WU and stretching activities. While WU mainly focuses on core body temperature elevation, stretching involves movements that stretch the muscle in order to increase the range of motions of joints or group of joints \cite{knudson2008warm}. 
Generally, WU procedures can be classified into active and passive WU procedures, and are centered on increase in core and muscle temperature. However, active and passive WU accomplish this objective through different approaches. The former involves raising muscle or core temperature by some external means (e.g. hot showers, saunas), while the latter aims to increase the body temperature through active movements of the major muscle groups (e.g. jogging, cycling, swimming) \cite{shellock1985warming, bishop2003warm2}. The most effective WU that could potentially effect the subsequent performance generally depends on the duration, intensity and the nature of the sports activity to be performed \cite{bishop2003warm2}. As each sport has its own unique requirements, it is difficult to specify a general WU routine that is beneficial and have positive impact by maximizing the subsequent performance. Nonetheless, it is suggested that a proper WU should use general, whole-body movements and last 5-10 minutes which is followed by a 5 minutes recovery period \cite{bishop2003warm2}.  
%dodaj za fatigue
%Several studies were conducted in the 1950s-1970s to investigate the effects of warming-up on athletic performance
%(Richards, 1968). In this context, approximately 60% of these studies found that warm-up was better
%to perform than no warm-up, whereas ~11% found that no warm-up was better, and the remaining ~29% found
%no significant differences between different protocols of warm-up and no warm-up (Blank, 1955). 
%tu sad das ove linkove
%(Generally, a warm-up to minimize impairments and enhance performance should be composed of a submaximal intensity aerobic activity followed by large amplitude dynamic stretching and then completed with sport-specific dynamic activities.
%these say that some stretching is ok
%http://www.jospt.org/doi/pdf/10.2519/jospt.1994.19.1.12
%https://www.ncbi.nlm.nih.gov/pubmed/21373870
%The efficacy, and characteristics, of warm-up and re-warm-up practices in soccer players: a systematic review. This review demonstrated that a static stretching WU reduced acute subsequent performance, while WU activities that include dynamic stretching, PAP-based exercises, and the FIFA 11+ can elicit positive effects in soccer players. The efficacy of an active RWU during half-time is also justified.
%ovo se placa nesto 
 %http://greatist.com/fitness/stretching-dynamic-warmup-040413
\\Although, considering the aforementioned benefits, is widely recommended to undertake the practice of WU, many amateur and recreational athletes do not seem to perform a proper WU before an exercise \cite{fradkin2010effects}. The reasons for this are manifold. Some people do not realize the importance of WU, find it tiresome or pressed for time and eager for instantaneous results, start with the more strenuous activity immediately. A recent survey carried out by Fradkin, \textit{et al.} including 1040 golfers and their WU habits, revealed the most common reasons for not warming-up. The survey showed that out of all the questioned golfers, over 70\% never or rarely warm-up. The most common reasons for not performing a proper WU routine were the perception that WU is needless (38.7\%), lack of time (36.4\%) and that they do not want to be bothered with this routine (33.7\%).
% A survey of 1040 randomly selected golfers was conducted over a 3-week period in July 1999. Information about golf participation, usual warm-up habits and reasons for these warm-up behaviours was obtained by a verbally administered self-report survey. Over 70% of the surveyed golfers stated that they never or seldom warm-up, with only 3.8% reporting warming-up on every occasion. The most common reasons why golfers warmed-up included to play better (74.5%), to prevent injury (27.0%), and because everyone else does (13.2%). Common reasons for not warming-up were the perception that they don't need to (38.7%), don't have enough time (36.4%) and can't be bothered (33.7%).
These results suggest that educational and motivational solutions with primary focus on benefits of WU, including injury prevention, need to be developed and implemented in order to increase the proportion of athletes who engage in WU routines before every strenuous exercise. One possible solution is the usage of \textit{gamification} in motivating athletes to perform WU more regularly. 
\pagebreak
\section{Gamification}
Having outlined the basic concepts regarding Warm-up procedures, the following section sheds light 
on the dimensions of Gamification. In order to tie in with the idea of comparing both concepts, after  introducing Gamification in further detail, the emphasis will be placed upon taking a look at the 
gamified solutions in the domain of fitness and health. 
\subsection{Introduction}
%http://link.springer.com/chapter/10.1007%2F978-3-319-07127-5_23
%(http://www.enterprise-gamification.com/mediawiki/index.php?title=Category:Gamification_Design_Elements)
% are commonly used 
%check this here https://badgeville.com/wiki/health
%% add gamification example reference
% http://www.enterprise-gamification.com/mediawiki/index.php?title=Gamification_Examples
In recent years, there have been an tremendous increase in popularity of video games inspired software solutions designed to address issues in variety of functional areas, incentivize consumer behavior or increase motivation and desire for achievement. What these software solutions all have in common is that they are based on the concept of \textit{gamification}. This term has begun to rise in popularity in 2010 (Figure \ref{fig:buzz}), and since then has been a trending topic. %ovde dodaj 
%It has proved to be an effective tool for certain businesses for developing new skills, solve problems, improve results or address.
\begin{figure}[h]
    \centering
    \includegraphics[width=\textwidth]{buzz}
    \caption{Google search frequency of the term \textit{gamification} from Janauary 2010 through January 2017. Data source: Google Trends, www.google.com/trends}
    \label{fig:buzz}
\end{figure}
Gamification is being used and studied in various domains, from education and academic performance to health care, finance, company culture building and recruitment, to name a few \cite{gamificationExamples, gamificationWiki, enterpriseGamify}. Large companies like Nike \cite{nikePlus}, Deloitte \cite{deloitte}, Starbucks \cite{starbucks}, Coca Cola \cite{coke} and Toyota \cite{toyota} have all used gamified solutions in order to increase customer loyalty, change behaviors, 
and drive innovation. %example u STUFF
Moreover, there is an increasing number of startups (e.g. Foursquare, CodeAcademy) that have gamification  at  their  core \cite{codeacademy} or offer assistance to enterprises to gamify their existing services (e.g. Badgeville \cite{badgeville}). Hamari \textit{et al.} reported on an increasing popularity of gamification related researches in the academia \cite{hamari2014does}. Figure \ref{fig:pub} gives an overview of the increase of writing on this topic. The figure includes only the number of publications for every year for the term \textit{Gamification} and excludes patents and citations. 
\begin{figure}[h]
    \centering
    \includegraphics[width=\textwidth]{pub}
    \caption{Search hits on 'Gamification'. Data source: www.scholar.google.com}
    \label{fig:pub}
\end{figure}
It is worth noticing that the appearance of term Gamification publication titles' has been increasing more rapidly than search hits for the same term (see Figure \ref{fig:buzz}). This suggests that Gamification is becoming more popular in academic circles as a research topic. 
\subsection{Behavioral  Psychology}
\subsection{Defining Gamification}
As  the  term  itself  is  relatively  new,  there  exist  numerous definitions  of  gamification  (Zicherman \&  Cunningham 2011, Kapp 2011, Werbach \& Hunter 2012). Definition by Deterding \textit{et al.} (2011) is currently the most cited definition of gamification in academia and is the definition that is adopeted for this thesis. In their paper the authors proposed a well reasoned definition as follows:
\begin{quotation}
\textit{``Gamification is the use of game design elements in a non-game context.''}
\end{quotation}
There exist references to \textit{gamifying} online systems as early as 1980. Professor Richard Bartle from University of Essex, points out the word referred originally to ``turning something not a game into a game.''\cite{werbach2012win}%ovo je knjiga, daj stranu 
 However, the first use of gamification in its current sense dates back to 2002 by Nick Pelling as part of his consultancy business, but the term did not see widespread adoption before the second half of 2010 \cite{marczewski2013gamification}. In parallel with this term, a verb \textit{to gamify} emerged. It's meaning reffers to applying game mechanics to supercharge user engagement, loyalty and fun \cite{toGamify}. 
It should be noted that the definition outlined by Deterding \textit{et al.} relates to \textit{games} and not \textit{play} \cite{deterding2011game}. %Consequently, the definition distinguieshes between \textit{gamefullness} and \textit{playfullness} ... TODO. 
Even though often used interchangeably, there exists a complex relationship between these two concepts and clear distinction can be made. That is, according to the forms they take in the world, \textit{play} can be interpreted as a broader category that includes \textit{game} as a subset \cite{salen2004rules}. Play is normally assumed to be a free-form activity lacking constraints engaged in for pleasure and amusement rather than a serious or practical purpose, whereas games provide context for actions and are limited in action by fixed rules \cite{juul2011half}. In addition, Salen \& Zimmerman (2004) define game as a system where players engage in an artificial conflict which is defined by rules that limit players behavior and define the game, that can result in a quantifiable outcome or goal \cite{salen2004rules}.%behavioral, game-playing
Games manifest themselves as integrated experiences, but they are built from many smaller pieces often called game elements \cite{werbach2012win}. They represent parts of games used as a building blocks for creating gamified applications, as well as tools and rules that define the overall context of game \cite{gamDesElem}. This means that the definition distinguishes gamification from other systems that employ full-fledged games rather than elements of game design only. Furthermore, it does not include all game elements either, only a subcategory called game design elements that are used as seen the most suitable in current situation. %ova dve poslednje recenice odavde, izmeni 
%http://ludus.hu/en/gamification/
The final aspect of the definition is that gamification operates in non-game contexts. A non-game context refers to applications which main purpose is beyond pure entertainment; using game design elements to a context of ``other than games''. This implies that gamification can be used and successfully applied to almost anything: from business, finance, personal improvement to education, health and fitness \cite{deterding2011game}. Thus, the challenge of gamification, is to select elements that normally operate within the game universe and apply them effectively in the real world.
%Although it is possible to also use gamification in the context  of  games  and  gamify  them,  it  can  simply  be  considered  to  be  a  part  of  designing  a  game.  This  way  specifying  gamification  to  nongame  contexts  is justified. (Deterding et al. 20
The concept of gamification is closely related to similar pre-existing concepts such as serious games, playful design and toys. Thus, the proposed definition aims at separating the concept of gamification from similar phenomena on a two-by-two matrix introduced by Deterding \textit{et al} (2011, Figure \ref{fig:mesh1}). 
\begin{figure}[h]
    \centering
    \includegraphics[width=0.75\textwidth]{gamification-btw-game-and-play}
    \caption{The matrix distinguishing the concepts related to gamification}
    \label{fig:mesh1}
\end{figure}
In figure \ref{fig:mesh1}, along one axis a distinction between gaming and playing is made, and on the other between whole game and an artifact with game elements. Gameful design or gamification, differs from playful design because the former focuses on activities that are goal oriented and structured by rules, while the latter focuses on activities that are based on improvisation and are free of form. Moreover, gamification is situated in the quadrant involving games and game elements, meaning that gamification makes use of gameful design rather than playful design and game elements rather than full-fledged games. This is different to serious games used also in non-game contexts, a group that includes full games that have been created for reasons other than pure entertainment. 
\subsection{Classification of Game elements}
% \cite{schobel2016agony}.% Sch{\"o}bel \textit{et al.} carried out a literature review to analyze the gamification elements used in various research studies.
%prezentacija o game vs play http://gamification-research.org/2012/04/defining-gamification/
%detering definise elements of game design. five leveles.Next the most common
%Next, and overview and various classification frameworks of game elements which might further enhance engagement, the potential goal of gamification is presented. 
Tools and rules that define the overall context of game are known as game elements. %from wiki Games
Gamified  applications  make use of game elements that in order to provide a gamified expirience and not to  give  rise entire (full-fledged) games \cite{deterding2011game}. With respect to the use of game elements, gamification studies classified them differently. Derived from the available literature, Deterding \textit{et al.} found that game elements previously identified and presented in different research studies, fell in five distinct levels of abstraction. Table \ref{table:gameElements} present a model for classification of game elements with five levels of abstraction, ordered from concrete to abstract.

% Please add the following required packages to your document preamble:
% \usepackage[normalem]{ulem}
% \useunder{\uline}{\ul}{}
\begin{table}[!htbp]
\centering
\caption{Taxonomy of game design elements by level of abstraction by Deterding \textit{et al.} (2011)}
\label{table:gameElements}
\begin{tabular}{lll}
\hline
\textbf{Level} & \textbf{Description} & \textbf{Example} \\ \hline
\begin{tabular}[c]{@{}l@{}}Game interface\\ design patterns\end{tabular} & \begin{tabular}[c]{@{}l@{}}Common, successful interaction\\  design components and design \\ solutions for a known problem\\  in a context, including prototypical\\  implementations\end{tabular} & \begin{tabular}[c]{@{}l@{}}Badge, leaderboard, \\ level\end{tabular} \\ \hline
\begin{tabular}[c]{@{}l@{}}Game design\\ patterns and\\ mechanics\end{tabular} & \begin{tabular}[c]{@{}l@{}}Commonly reoccurring parts of \\ the design of a game that\\  concern gameplay\end{tabular} & \begin{tabular}[c]{@{}l@{}}Time constraint, \\ limited resources, turns\end{tabular} \\ \hline
\begin{tabular}[c]{@{}l@{}}Game design\\ principles and\\ heuristics\end{tabular} & \begin{tabular}[c]{@{}l@{}}Evaluative guidelines to approach a\\  design problem or analyze a given\\ design solution\end{tabular} & \begin{tabular}[c]{@{}l@{}}Enduring play,\\ clear goals, \\ variety of game styles\end{tabular} \\ \hline
Game models & \begin{tabular}[c]{@{}l@{}}Conceptual models of the components of\\  games or game experience\end{tabular} & \begin{tabular}[c]{@{}l@{}}
MDA; challenge, \\ fantasy, curiosity;\\ game design atoms; \\ Core Elements of the \\ Gaming Experience \end{tabular} \\ \hline
\begin{tabular}[c]{@{}l@{}}Game design\\ methods\end{tabular} & \begin{tabular}[c]{@{}l@{}}Game design-specific\\  practices and processes\end{tabular} & \begin{tabular}[c]{@{}l@{}}Playtesting,\\ playcentric design, \\ value conscious\\ game design\end{tabular} \\ \hline
\end{tabular}
\end{table}
On the other hand, Zichermann \& Cunningham (2011) take a different approach and base their description of game elements on the MDA framework, which is categorized as a \textit{game model} in the framework proposed by Deterding \textit{et al.}(2011). It is one of the most frequently used frameworks of game design and stands for \textit{Mechanics}, \textit{Dynamics} and \textit{Aesthetics} \cite{hunicke2004mda}. 
Mechanics are the functioning components that make up the game. They represent the specific elements of the game and the different behaviors and control mechanism that are given to the player within the game's context. Dynamics, on the other side, represent the player’s interactions with the mechanics. They specify how the player reacts to the mechanics of the system, both individually and with other players. Lastly, the aesthetics of the system are the emotional reponses of the user who interact with the game system. \cite{zichermann2011gamification}.

In the next section different mechanics are listed and described according to their relevance and applicability to our context, considering also the  dynamics  and desirable aesthetic outcomes

\subsubsection{Game elements in gamified systems}
%just for health and fitness gamification
%First  the  elements  most related   to   motivation   are   presented   and   then   the   most   popular elements   in gamification  literature  are  presented.  Although  all  game  elements  can be  used  in gamification they might not all be equally relevant. Some are more potent and useful than others. As it is also a goal of this paper to examine which elements are the most  relevant,  based  on  the  research  done  on  motivational  theory  and  gamification literature a distinction is made.
%In theory, any context, task or process can be gamified \cite{muntean2011raising}. %skini i procitaj Muntean!! 
Although all game elements can be used in gamified systems they might not all be equally relevant. Some are more potent and useful than others.%ovu recenicu sa potentom izbrisi.

\subsection{Objectives}
\subsection{Principles}
\subsection{The impact of Gamification}
\subsubsection{Gamification in sports - Effects and Objectives}
%The main goal of gamification is to engage the users Gamification’s main goal is to rise the engagement of users by using game-like techniques such asscoreboards and personalized fast feedback (Flatla et al, 2011) making people feel more ownership and purpose when engaging with tasks (Pavlus, 2010).\cite{burke2016gamify}
\subsection{Gamification critiques}
\subsection{Gamification of Health and Fitness}
%Gamification in Health and Fitness has rapidly emerged over the past decade as a tool to promote health and wellness. It is a broad term referring to the use of game thinking and game mechanics in a non-game context to engage users and solve problems. The concept is used to incentivise users to achieve their goals and increase user engagement. The best examples of gamification are in the Health and Fitness industry, where games encourage exercise by turning physical activity into a game and by delivering health interventions for bad habits cessation, like smoking, overeating or poor hydration, and medication adherence. Application of mobile and wearable devices have proven to be effective platforms for health and fitness games due to its wide adaptation, ease of use and continuous proximity to the users and patients. Since gamification can be applied to almost any business model, serious or not, for this thesis, we restrict our concern of gamification to the solving of serious issues. In particular, we are focused on gamification of education and behavior change related to the serious world issue of childhood obesity

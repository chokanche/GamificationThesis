\chapter{Literature review}\label{chapter:warmup}

\section{Introduction}

The following chapter provides an overview of the conceptually related works in the domains of warm up and gamified solutions relevant to fitness and exercise. 
To impose structure, first an overview of studies and results regarding benefits of warm up prior performing exercise is given. Following, the concept of gamification and most relevant ... in ... At the end of this chapter, a comparison between the presented approaches and our solution is made.
\section{Warm up in sports}
Despite very contrasting believes and limited scientific evidence regarding it's effectiveness in many situations, warm-up (WU) has become a standard practice among professional and
recreational athletes \cite{bishop2003warm1, bishop2003warm2, shellock1985warming}. WU in sports is defined as a period of preparatory exercise which is carried out in order to prepare the athlete for the demands of the subsequent physical activity \cite{karvonen1992importance, woods2007warm, hedrick1992exercise}.
%karvonen skini i procitaj
Typically, WU includes a short and low-intensity preparatory activity which is followed by a stretching routine and sport specific exercise \cite{safran1989warm}. 
The purpose of WU is to enhance the subsequent competition or training performance and improve muscle dynamics to reduce the risk of sport-related injury\cite{bishop2003warm1, shellock1985warming, knudson2008warm}. 
 %HERE MENTION THE STUDIES THAT SAY THIS DOESNT HELP.  CHECK LEMON ILIEV
Nonetheless, there is still deficiency of scientific  evidence on what kind of WU can influence both muscle damage prevention and performance improvement \cite{safran1989warm}.\\ %cite
Fradkin, \textit{at al.} (2010) carried out a systematic review and meta-analysis of relevant studies concerning benefits of WU on the performance. They found that an adequate WU supports an improvement in performance in 79\% of the research studies analyzed. Furthermore, they pointed out that there exist little evidence supporting detrimental effects WU might have on performance and sports participants.
WU can affect the performance via variety of temperature and non-temperature related mechanisms \cite{bishop2003warm1}. 
%By performing a low intensity training routine before taking part in more demanding exercise, an increase in one's body temperature \(Magnusson 
%et al., 2000\) and muscle blood flow occurs \(Tiidus \& Shoemaker, 1995\).  
The most relevant effects of WU can be attributed to physiological mechanisms like increased muscle temperature, decreased resistance of muscle and joints (decreased stiffness), increased oxygen delivery to muscles, increased nerve-conduction rate and speeding of metabolic reactions \cite{bishop2003warm1}. 
However, the benefits of WU are not exclusively physical. Apart from the physiological changes a body undergoes during this preparatory period, it has been hypothesized that a possible psychological benefit can also be gained by following a proper WU routine \cite{bishop2003warm1,shellock1985warming}.
It has been suggested that WU can serve as preparatory phase providing time for athlete to concentrate and mentally prepare for the forthcoming exercise \cite{shellock1985warming}. 
%Thus, possible psychological benefits is increased mental preparedness for the forthcoming exercise\cite{bishop2003warm1}. 
For instance, in the study that investigated the link between a WU and psychological processes, Ladvig (2013) \textbf{found} that athletes who reported \textbf{using} WU before engaging in more demanding physical activity demonstrated significantly higher levels of exercise related motivation and enjoyment. Thus increased motivation and enjoyment is an additional psychological benefit of WU \cite{ladwig2013psychological}.
 %findings of this theisis: http://aut.researchgateway.ac.nz/bitstream/handle/10292/325/WeerapongP.pdf?sequence=1
\\Apart from physiological and psychological benefits, WU has been suggested to have an important role in sports-related injury prevention \cite{shellock1985warming}. Unfortunatelly, there exist no high-quality research studies in order to draw definite conclusion as to the effect WU has on sports-related injury prevention \cite{fields2007should}. Safran, \textit{et al.} proposed a possible bio-mechanical explanation for injury reduction with WU. The results of this study reported that warmed-up muscles in the animal models can elongate more before failure caused by increased force and length of stretch \cite{safran1989warm}. In the study of Fradkin, \textit{et al.} the current evidence regarding WU in injury prevention has been assessed. Out of five high-quality studies with sufficient data that have been systematically reviewed, three studies reported significant injury related reduction by performing WU before the physical activity while in other two no benefits were reported. Overall, based on the weight of evidence, it has been concluded in favor of WU to decrease the risk of injury.\\
%Furthermore, Nosaka and Clarckson found that high and low intensities of WU could reduce the magnitude musculatory damage ... They proposed that 
%A search of the literature identified only one published research paper on the effects of warm-up on the severity of muscle damage (Nosaka & Clarkson, 1997).  Nosaka and Clarkson (1997) found that both high (100 repetitions of maximal concentric contraction) and low (100 repetitions of minimal concentric contraction) intensities of 
%warm-up could reduce the magnitude of mu
%scle damage as indicated by reduced 
%soreness sensation, strength and range of 
%motion loss, swelling, and creatine kinase 
%activity.  The authors proposed that warm-up 
%might help to increase muscle temperature 
%and circulation, and consequen
%tly, increase muscle and conne
%ctive tissue el
%asticity
%the majority of effects of warm up have been attributed to temperature related mechanism
There exist various types of WU procedures professional and recreational athletes use as a preparatory phase for the physically more demanding exercise preparation. It is important to distinguish between WU and stretching activities. While WU mainly focuses on core body temperature elevation, stretching involves movements that stretch the muscle in order to increase the range of motions of joints or group of joints \cite{knudson2008warm}. 
Generally, WU procedures can be classified into active and passive WU procedures, and are centered on increase in core and muscle temperature. However, active and passive WU accomplish this objective through different approaches. The former involves raising muscle or core temperature by some external means (e.g. hot showers, saunas), while the latter aims to increase the body temperature through active movements of the major muscle groups (e.g. jogging, cycling, swimming) \cite{shellock1985warming, bishop2003warm2}. The most effective WU that could potentially effect the subsequent performance generally depends on the duration, intensity and the nature of the sports activity to be performed \cite{bishop2003warm2}. As each sport has its own unique requirements, it is difficult to specify a general WU routine that is beneficial and have positive impact by maximizing the subsequent performance. Nonetheless, it is suggested that a proper WU should use general, whole-body movements and last 5-10 minutes which is followed by a 5 minutes recovery period \cite{bishop2003warm2}.  
%dodaj za fatigue
%Several studies were conducted in the 1950s-1970s to investigate the effects of warming-up on athletic performance
%(Richards, 1968). In this context, approximately 60% of these studies found that warm-up was better
%to perform than no warm-up, whereas ~11% found that no warm-up was better, and the remaining ~29% found
%no significant differences between different protocols of warm-up and no warm-up (Blank, 1955). 
%tu sad das ove linkove
%(Generally, a warm-up to minimize impairments and enhance performance should be composed of a submaximal intensity aerobic activity followed by large amplitude dynamic stretching and then completed with sport-specific dynamic activities.
%these say that some stretching is ok
%http://www.jospt.org/doi/pdf/10.2519/jospt.1994.19.1.12
%https://www.ncbi.nlm.nih.gov/pubmed/21373870
%The efficacy, and characteristics, of warm-up and re-warm-up practices in soccer players: a systematic review. This review demonstrated that a static stretching WU reduced acute subsequent performance, while WU activities that include dynamic stretching, PAP-based exercises, and the FIFA 11+ can elicit positive effects in soccer players. The efficacy of an active RWU during half-time is also justified.
%ovo se placa nesto 
 %http://greatist.com/fitness/stretching-dynamic-warmup-040413
\\Although, considering the aforementioned benefits, is widely recommended to undertake the practice of WU, many amateur and recreational athletes do not seem to perform a proper WU before an exercise \cite{fradkin2010effects}. The reasons for this are manifold. Some people do not realize the importance of WU, find it tiresome or pressed for time and eager for instantaneous results, start with the more strenuous activity immediately. A recent survey carried out by Fradkin, \textit{et al.} including 1040 golfers and their WU habits, revealed the most common reasons for not warming-up. The survey showed that out of all the questioned golfers, over 70\% never or rarely warm-up. The most common reasons for not performing a proper WU routine were the perception that WU is needless (38.7\%), lack of time (36.4\%) and that they do not want to be bothered with this routine (33.7\%).
% A survey of 1040 randomly selected golfers was conducted over a 3-week period in July 1999. Information about golf participation, usual warm-up habits and reasons for these warm-up behaviours was obtained by a verbally administered self-report survey. Over 70% of the surveyed golfers stated that they never or seldom warm-up, with only 3.8% reporting warming-up on every occasion. The most common reasons why golfers warmed-up included to play better (74.5%), to prevent injury (27.0%), and because everyone else does (13.2%). Common reasons for not warming-up were the perception that they don't need to (38.7%), don't have enough time (36.4%) and can't be bothered (33.7%).
These results suggest that educational and motivational solutions with primary focus on benefits of WU, including injury prevention, need to be developed and implemented in order to increase the proportion of athletes who engage in WU routines before every strenuous exercise. One possible solution is the usage of \textit{gamification} in motivating athletes to perform WU more regularly. 

\subsection{Gamification}
In the recent years, there has been an tremendous increase in popularity of
video games inspired software solutions (cite maybe) which main goals are to
develop new skills, solve problems or address issues in a variety of
functional areas. What these software solutions all have in common is that they
are based on the concept of \textit{gamification}, defined by Detering et al. in  as the
use of game design elements characteristic for games in a non-game contexts to solve a problem or engage the audience. 
It is a tool to change the behavior through positive reinforcement. 
%izmemi ovu recenicu
The main idea of gamification is to make tasks XX tasks  by using elements
known from games
(%http://link.springer.com/chapter/10.1007%2F978-3-319-07127-5_23
) (e.g. receiving points) and thereby provide
incentives that make fulfilling the task more rewarding. W
%(http://www.enterprise-gamification.com/mediawiki/index.php?title=Category:Gamification_Design_Elements)
% are commonly used 
Up until now, gamification has been researched in various domains. For example,
in the domain of teaching and education (%http://www.nature.com/nbt/journal/v32/n7/full/nbt.2955.html
), marketing (), business
(), technology design (), environmental behaviour (), crowdsourcing (). Apart
from the mentioned, probably the most prevalent one is the area of health and exercise (fitness). 
%check this here https://badgeville.com/wiki/health
why?

%% add gamification example reference
% http://www.enterprise-gamification.com/mediawiki/index.php?title=Gamification_Examples
\chapter{Literature review}\label{chapter:relatedwork}
%Deterding et al. go on to explain that a gamified system, like a game, usually has rules and is goal oriented. Gamification is not a full-fledged game, although it utilizes many techniques that games implement such as, levels, clear goals, time constraints, badges, value conscious game design, challenge, limited resources and leader boards. It allows people to stay grounded in reality, whilestill profiting from successful game benefits like gaining access to a person’s emotions and intrinsic motivation, which in turn will help create habits. ZA ABRSTACT??
The following chapter provides an overview of the past research related to \acrfull{wu}, a preparatory exercise performed prior to physical activity, theories that explain players' engagement and motivation when interacting with a gamified system, and conceptually related works in the domain of commercial and non-commercial gamified solutions relevant to fitness and exercise. In other words, this section explores the overlap of five different fields such as exergames, gamification, health and fitness, injury prevention and performance improvement, and motivational psychology. This concept is illustrated in Figure \ref{fig:diagram}.
\begin{figure}[h]
    \centering
    \includegraphics[width=0.60\textwidth]{diagram}
    \caption{The five general fields that relate to this thesis. The target field is presented as the overlap of the five fields.}
    \label{fig:diagram}
\end{figure}\\
To impose structure, the basic concepts related to \acrshort{wu} and an overview of studies regarding benefits of \acrshort{wu} are given. Following, the concepts of gamification and exergames are introduced. Finally, an overview of theories that describe and explain various psychological effects that games have on players is presented together with the direction the research in this thesis will take and the motivations behind it.
\section{Warm Up in Sports}
\subsection{The Importance of Physical Activity}
In the last few decades there has been a significant increase of women and men who engage in some sort of physical activity. Physical activity is beneficial to ones health. According to medical professionals regular physical activity can significantly decrease the commonness of chronic diseases such as high blood pressure, heart disease,
(colon and breast) cancer, hypertension and diabetes as well as reduce cardiovascular-related deaths, to name a few \cite{mayr2015prevention, warburton2006health}. Regular exercise reduces the incidence of obesity
and obesity-related illnesses, maintains a general standard of health and is associated with a reduced risk of premature death \cite{warburton2006health}. Moreover, regular engagement in sports of any kind can also serve as a countermeasure for psychological disorders and greatly limit the severity of
episodes of anxiety and depression \cite{mayr2015prevention}.
\subsection{Overview of Sports Injury}\label{subsection:injury}
The counterargument to all of the mentioned health benefits is that engaging in regular physical activity is often associated with a higher risk of injury which can occur in athletes of all age \cite{van1997severity}. To put it differently, there exists a higher risk of injury of the musculoskeletal system including soft tissue
damage, fractures, ligament and tendon tears, and nerve injuries in athletes who engage in some sort of physical (sport) activity \cite{mayr2015prevention}. Sport related injuries generally occur in joints: the knee, ankle, hip, shoulder, elbow, wrist and spine, usually from a sports related accident but often due to overuse, repetitive microtraumas that are solely insufficient to cause macroscopic injuries
 \cite{mayr2015prevention}. During international athletics championships between 2007 and 2014, data regarding injuries has been collected in order to compare the characteristics of injuries between female and male athletes \cite{edouard2015sex}. The results showed that males suffered more thigh strains than female athletes and that injury incidences differed between genders for location, type, and event groups. The results concerning main injury locations for female and male athletes are presented in Figure \ref{fig:injuries}. The type of the injury depends on many factors and usually is divided into: \textit{intrinsic} and \textit{extrinsic} \cite{mayr2015prevention, lefevre2016major}. \\ Generally, extrinsic injuries are linked to the practice of sports itself and the environment the activity is carried out. On the other hand, intrinsic injuries are tied to biological characteristics, anatomical factors, gender, and age, among others \cite{mayr2015prevention}. Other sports specialists group sport related injuries differently. For instance, \cite{fischer2016causes} differentiate between \textit{damage} as an overuse injury and \textit{injury} as an acute injury. They further state that an injury occurs in a single acute action (acute injury), while damage appears after repeated action as the result of many repetitive minor insults (overuse injury) \cite{fischer2016causes}. Addtitionally, \cite{pecina1993overuse} state how ``\textit{the main characteristic of an injury is acuteness, whereas damage has a chronic character}''. \\Acute injuries are more likely to occur in sports that include high-speed running, rapid movement, or full-body contact, whereas aerobic low-contact sports that include long training sessions may produce overuse injuries  \cite{mayr2015prevention}.
\begin{figure}[h]
    \centering
    \includegraphics[width=\textwidth]{injuries}
    \caption{Main injury location for female and male athletes during
international athletics championships from 2007 to 2014. Adapted from \cite{mayr2015prevention}}
    \label{fig:injuries}
\end{figure}\\
In order to prevent injuries, athletes should receive the correct amount of training and recovery period, and have a healthy lifestyle. The correct amount of training depends on the type of the physical activity itself, as much as the physical characteristics of the athlete. Moreover, the sports technique must be correct and a good quality equipment that is adapted to the player (morphology and level of play) must be used in order to prevent injuries. Injury prevention strategies should be gender-specific. That is, as discussed in \cite{edouard2015sex} and presented in Figure \ref{fig:injuries}, for injury prevention ``\textit{one size does not fit all}'', and hence it should be adapted to the differences in injury characteristics between female and male athletes. Lastly, as one of the main sports injury prevention mechanism different studies outline that every physical activity must be preceded by a suitable \acrshort{wu} procedure. This preparatory activity is hypothesized to give athletes sufficient time to adjust and prepare for a more intense susbsequent activity thereby reducing the likelihood of injuries \cite{mayr2015prevention}.\\\\ Next, a definition and an overview of \acrshort{wu} prior sports activities, together with its types and major benefits is presented.\pagebreak %NISI BAS PUNO O WARM UP + INJURY PREVENTION PISAO OVDE and should end with a cooling down phase . SVE ovo imas u onoj knjizi ove poslednje reference. mozda dodas nesto jos??  
\subsection{Defining Warm Up}
Despite very contrasting beliefs and limited scientific evidence regarding its effectiveness in many situations, WU has become a standard practice among professional and recreational athletes \cite{bishop2003warm1, bishop2003warm2, shellock1985warming}. WU in sports is defined as a period of preparatory exercise which is carried out in order to prepare the athlete for the demands of the subsequent physical activity  \cite{karvonen1992importance, woods2007warm, hedrick1992exercise}.
%karvonen skini i procitaj
Typically, WU includes a short and low-intensity preparatory activity which is followed by a stretching routine and sports specific exercise \cite{safran1989warm}. 
The ideal WU depends on the physical activity performed,
the level of competition, and the age of the participants. Moreover, the ideal WU should include the muscle groups that are required during the training or competition
 \cite{mayr2015prevention}. Various studies point out that the main purpose of WU is to enhance the subsequent competition or training performance and improve muscle dynamics to reduce the risk of sport-related injury \cite{bishop2003warm1, shellock1985warming, knudson2008warm}. 
 %HERE MENTION THE STUDIES THAT SAY THIS DOESNT HELP.  CHECK LEMON ILIEV
Nonetheless, there is still deficiency of scientific evidence on what kind of WU can influence both muscle damage prevention and performance improvement \cite{safran1989warm}.\\\\ The following section will gives an overview of some of the assumed benefits of WU as a preparatory routine before physically more demanding exercise. 
\subsection{The Benefits of Warm Up}
In a systematic review and meta-analysis of relevant studies concerning the benefits of WU on performance, \cite{fradkin2006does} found that an adequate WU supports an improvement in performance in 79\% of the research studies analyzed. Furthermore, they pointed out that there exists little evidence supporting detrimental effects WU might have on performance and sports participants.
WU can affect the performance via variety of temperature and non-temperature related mechanisms \cite{bishop2003warm1}. 
%By performing a low intensity training routine before taking part in more demanding exercise, an increase in one's body temperature \(Magnusson 
%et al., 2000\) and muscle blood flow occurs \(Tiidus \& Shoemaker, 1995\).  
The most relevant effects of WU can be attributed to physiological mechanisms like increased muscle temperature, decreased resistance of muscle and joints (decreased stiffness), increased oxygen delivery to muscles, increased nerve-conduction rate and speeding of metabolic reactions \cite{bishop2003warm1}. 
However, the benefits of WU are not exclusively physical. Apart from the physiological changes a body undergoes during this preparatory period, it has been hypothesized that a possible psychological benefit can also be gained by following a proper WU routine \cite{bishop2003warm1,shellock1985warming}.
It has been suggested that WU can serve as a preparatory phase providing time for athletes to concentrate and mentally prepare for the forthcoming exercise \cite{shellock1985warming}. 
%Thus, possible psychological benefits is increased mental preparedness for the forthcoming exercise\cite{bishop2003warm1}. 
Moreover, in the study that investigated the link between a WU and psychological processes \cite{ladwig2013psychological}, it has been reported that athletes who performed a proper WU routine before engaging in more demanding physical activity demonstrated significantly higher levels of exercise related motivation and enjoyment. Thus, increased motivation and enjoyment is an additional psychological benefit of WU.
 %findings of this theisis: http://aut.researchgateway.ac.nz/bitstream/handle/10292/325/WeerapongP.pdf?sequence=1
\\\\Apart from physiological and psychological benefits, WU has been suggested to have an important role in sports-related injury prevention \cite{shellock1985warming}. Unfortunately, there exist no high-quality research studies in order to draw a definite conclusion as to the effect of WU on sports-related injury prevention \cite{fields2007should}. In \cite{fradkin2006does} the researchers reviewed five high-quality studies that investigated the effects of warming up in humans on injury risk in physical activity. Five studies reported sufficient data on the effects of warming up on reducing injury risk in humans. However, only three of the studies found that performing a WU prior to performance significantly reduced the risk of injury in athletes, while the remaining two found that warming up has no effects in injury decrease \cite{fradkin2006does}. Therefore, the researchers concluded that there is insufficient evidence to endorse or discontinue WU routine prior to physical activity in order to prevent injury among sports participants. However, the weight of evidence is in favor of a decreased risk of injury. A possible bio-mechanical explanation for injury reduction with WU has been presented in \cite{safran1989warm}. The results of this study showed that warmed-up muscles in the animal models can elongate more before failure caused by increased force and length of stretch.
%Furthermore, Nosaka and Clarckson found that high and low intensities of WU could reduce the magnitude musculatory damage ... They proposed that 
%A search of the literature identified only one published research paper on the effects of warm-up on the severity of muscle damage (Nosaka & Clarkson, 1997).  Nosaka and Clarkson (1997) found that both high (100 repetitions of maximal concentric contraction) and low (100 repetitions of minimal concentric contraction) intensities of 
%warm-up could reduce the magnitude of mu
%scle damage as indicated by reduced 
%soreness sensation, strength and range of 
%motion loss, swelling, and creatine kinase 
%activity.  The authors proposed that warm-up 
%might help to increase muscle temperature 
%and circulation, and consequen
%tly, increase muscle and conne
%ctive tissue el
%asticity
%the majority of effects of warm up have been attributed to temperature related mechanism
%TO DISCUSS STRETCHING! HERE %
%ne znam da li je ovo za komponenete dobro ovde? mozda bi trebao dodati jos nesto..
\subsection{Types of Warm Up}
There exist various types of WU procedures that professional and recreational athletes at any level use as a preparatory phase before the physically more demanding exercise. According to \cite{safran1989warm}, an appropriate WU procedure should consist of three factors. These factors represent the WU components mentioned most often in the sports literature. However, recent studies question the importance and appropriateness of stretching as a component of a proper WU procedure \cite{pereles2012large}. The components are as follows:
\begin{itemize}
\item a period of aerobic exercise to increase body
temperature \cite{safran1989warm},
\item a period of sport-specific stretching to stretch 
the muscles to be used in the subsequent
performance \cite{safran1989warm} and
\item a period of activity incorporating movements
similar to those to be used in the subsequent
performance \cite{safran1989warm}.
\end{itemize}  
First, it is important to distinguish between WU and stretching activities. While WU mainly focuses on core body temperature elevation, stretching involves movements that stretch the muscle in order to increase the range of motions of joints or group of joints \cite{knudson2008warm}. 
Generally, WU procedures can be classified into \textit{passive} and \textit{active} WU procedures and are centered on increase in core and muscle temperature. They accomplish this objective through different approaches. The former involves raising muscle or core temperature by some external means (e.g. hot showers, saunas), while the latter aims to increase the body temperature through active movements of the major muscle groups (e.g. jogging, cycling, swimming) \cite{bishop2003warm2, shellock1985warming}. \\The most effective WU that could potentially affect the subsequent performance generally depends on the duration, intensity, and the nature of the sports activity to be performed \cite{bishop2003warm2}. As each sport has its own unique requirements, it is difficult to specify a general WU routine that is beneficial and has a positive impact by maximizing the subsequent performance. Nonetheless, it is suggested that a proper WU should use general, whole-body movements and last five to ten minutes, followed by a five minutes recovery period \cite{bishop2003warm2}. However, in cold weather, the duration of the WU procedure should be increased \cite{mayr2015prevention}. %Vec si pisao o faktorima wu prethodno, pa ovde imas kontradikciju + You just mentioned performance, now introduce somehow injury prevention objasni zasto pises sad o injury prevention programu kad nema bas dokaza da ima efekta%
One example of WU procedure widely used in football which is easily adapted to other sports is the \acrfull{fifa}, developed in cooperation with national and international experts under the leadership of the \acrfull{fmarc}, in order to reduce the incidence of football injuries and maximize the subsequent performance \cite{fifa}. The program includes various exercises that focus on core stabilization, and eccentric training of thigh muscles, to name a few. A recent review  \cite{barengo2014impact} showed how the \acrshort{fifa} program can decrease the incidence of injuries in amateur football players and also improve neuromuscular performance, enough to consider this program a fundamental public health intervention.\\*\\*
%reci da cemo koristiti vezbe koje su recommneded u fifa. dodaj za fatigue
%Several studies were conducted in the 1950s-1970s to investigate the effects of warming-up on athletic performance
%(Richards, 1968). In this context, approximately 60% of these studies found that warm-up was better
%to perform than no warm-up, whereas ~11% found that no warm-up was better, and the remaining ~29% found
%no significant differences between different protocols of warm-up and no warm-up (Blank, 1955). 
%tu sad das ove linkove
%(Generally, a warm-up to minimize impairments and enhance performance should be composed of a submaximal intensity aerobic activity followed by large amplitude dynamic stretching and then completed with sport-specific dynamic activities.
%these say that some stretching is ok
%http://www.jospt.org/doi/pdf/10.2519/jospt.1994.19.1.12
%https://www.ncbi.nlm.nih.gov/pubmed/21373870
%The efficacy, and characteristics, of warm-up and re-warm-up practices in soccer players: a systematic review. This review demonstrated that a static stretching WU reduced acute subsequent performance, while WU activities that include dynamic stretching, PAP-based exercises, and the FIFA 11+ can elicit positive effects in soccer players. The efficacy of an active RWU during half-time is also justified.
%ovo se placa nesto 
 %http://greatist.com/fitness/stretching-dynamic-warmup-040413
Although considering the aforementioned benefits and the fact it is widely recommended to undertake the practice of WU, many amateur and recreational athletes do not seem to perform a proper WU before an exercise \cite{fradkin2010effects}. The reasons for this are manifold. Some people do not realize the importance of WU, find it tiresome or being pressed for time and eager for instantaneous results, start with the more strenuous activity immediately. A recent survey \cite{fradkin2010effects} which included 1040 golfers and their WU habits revealed the most common reasons for not warming-up. The survey showed that out of all the questioned golfers, over 70\% never or rarely warm-up. The most common reasons for not performing a proper WU routine were the perception that WU is needless (38.7\%), lack of time (36.4\%) and that they do not want to be bothered with this routine (33.7\%). These results suggest that educational and motivational solutions with primary focus on the benefits of WU, including injury prevention, need to be developed and implemented in order to increase the proportion of athletes who engage in WU routines before every strenuous exercise. One possible solution is the usage of \textit{Gamification} and \textit{Exergames} in motivating athletes to perform WU more regularly. 
\pagebreak
\section{Gamification and Exergames}
Having outlined the basic concepts of WU procedures, the following section discusses the dimensions of gamification and exergames. In order to tie in with the idea of linking these concepts with WU procedures, the emphasis will also be placed on understanding the fundamental aspects behind human's motivation and engagement (Section \ref{chapter:motivation}). %Subsequently, we elaborate the techniques, elements, and benefits of Gamification and Exergames.
\subsection{Video Games and Exercise}
At first glance, most people think of video games and exercise as two concepts that are polar opposites and cannot coexist together. Exercise and sports are usually associated with being physically active and burning calories. On the other hand, video games are often linked to activities that involve hours of sitting down by your self to play, for example, \textit{World of Warcraft}\footnote{Or even \textit{Flappy Birds}}. However, these two concepts can actually complement each other. In both instances, one is seeking to be better at the task being performed. For example, an athlete will try to improve its best running time, and the video gamer will strive to beat its best game score. The missing link, that connects these activities together, is given in a form of gamification which leverages people's natural desires for competition, socializing, learning, mastery, achievement, status, self-expression, and closure in order to encourage and motivate individuals to exercise more frequently and improve their overall health.
%http://link.springer.com/chapter/10.1007%2F978-3-319-07127-5_23
%(http://www.enterprise-gamification.com/mediawiki/index.php?title=Category:Gamification_Design_Elements)
% are commonly used 
%check this here https://badgeville.com/wiki/health
%% add gamification example reference
% http://www.enterprise-gamification.com/mediawiki/index.php?title=Gamification_Examples
\subsection{A Primer on Gamification}
In recent years, there has been a tremendous increase in popularity of video games inspired software solutions designed to address issues in a variety of functional areas, incentivize consumer behavior or increase motivation and the desire for achievement. What these software solutions all have in common is that they are based on the concept of gamification. This term began to rise in popularity in 2010 (Figure \ref{fig:buzz}), and since then has been a trending topic \footnote{Data source: Google Trends, www.google.com/trends}. Gamification is being used and studied in various domains, from education and academic performance to health care, finance, company culture building, and recruitment, to name a few \cite{gamificationExamples, gamificationWiki, enterpriseGamify}. Large companies like \textit{Nike} \cite{nikePlus}, \textit{Deloitte} \cite{deloitte}, \textit{Starbucks} \cite{starbucks}, \textit{Coca Cola} \cite{coke}, and \textit{Toyota} \cite{toyota} have all used gamified solutions in order to increase customer loyalty, change behaviors, and drive innovation. \pagebreak 
\begin{figure}[h]
    \centering
    \includegraphics[width=\textwidth]{buzz}
    \caption{Google search frequency of the term ``gamification'' from Janauary 2010 through January 2017.}
    \label{fig:buzz}
\end{figure}\\
Gamified solutions for sports activities are becoming popular and widely used also, and according to \cite{iosPopulatity} the consumer segment comprised of millennials \footnote{Also known as Generation Y. A demographic cohort born between 1980s and the mid-1990s to early 2000s \cite{mill}}, have the highest inclination towards iOS fitness mobile applications (Figure \ref{fig:iosApps}). 
\begin{figure}[h]
    \centering
    \includegraphics[width=0.85\textwidth]{iosApps}
    \caption{Random sample of 15.271 American iOS owners. Adapted from \cite{iosPopulatity}}
    \label{fig:iosApps}
\end{figure}\\
An example of gamified solution for sports and exercise is the \textit{Strava} application and website that uses gamification elements in order to enhance the experience of sport and connect athletes with similar sports affinities from around the world \cite{strava}. Moreover, there is an increasing number of startups \cite{foursquare, codeacademy} that have gamification at their core or offer assistance to enterprises to gamify their existing services \cite{badgeville}. \\The increasing popularity of gamification related researches in the academia has also been reported in \cite{hamari2014does}. Figure \ref{fig:pub} gives an overview of the increase of writing on this topic. It includes only the number of publications for every year for the term ``gamification'' and excludes patents and citations \footnote{Data source: www.scholar.google.com}. 
\begin{figure}[h]
    \centering
    \includegraphics[width=\textwidth]{pub}
    \caption{Search hits on term ``gamification''. Adapted from \cite{hamari2014does}}
    \label{fig:pub}
\end{figure}\\
It is worth noticing that the appearance of the term ``gamification'' in publication titles has been increasing more rapidly than search hits for the same term (Figure \ref{fig:buzz}). This suggests that gamification is becoming more popular in academic circles as a research topic.%[TODO]read fred's comments on this + change years and add reference to search hits 
\paragraph{Defining Gamification}
There exist references to \textit{gamifying} online systems as early as 1980. Professor Richard Bartle from University of Essex, points out the word referred originally to ``\textit{turning something not a game into a game}'' \cite{werbach2012win}. The first use of gamification in its current sense dates back to 2002 by Nick Pelling as part of his consultancy business. However, the term did not see widespread adoption before the second half of 2010 \cite{marczewski2013gamification}. In parallel with this term, a verb \textit{to gamify} emerged. Its meaning refers to applying game mechanics to supercharge user engagement, loyalty and fun \cite{toGamify}. As the term itself is relatively  new,  there exist numerous definitions of gamification \cite{deterding2011game, werbach2012win, kapp2012gamification}. A definition introduced by \cite{deterding2011game} is currently the most cited one in academia, and is the definition that is adopted for this thesis.
\pagebreak
\\In their paper \cite{deterding2011game}, the authors proposed a well reasoned definition as follows:
\begin{quotation}
\textit{``Gamification is the use of game design elements in a non-game context.''}
\end{quotation}
It should be noted that the definition outlined by the researchers relates to \textit{games} and not \textit{play} \cite{deterding2011game}. Even though often used interchangeably, and there exists a complex relationship between these two concepts, a clear distinction can be made. That is, according to the forms they take in the world, \textit{play} can be interpreted as a broader category that includes \textit{game} as a subset \cite{salen2004rules}. Play is normally assumed to be a free-form activity lacking constraints engaged in for pleasure and amusement rather than a serious or practical purpose whereas games provide context for actions and are limited in action by fixed rules \cite{juul2011half}. In addition, \cite{salen2004rules} define game as a system where players engage in an artificial conflict which is defined by rules that limit players' behavior and define the game that can result in a quantifiable outcome or goal. Games manifest themselves as integrated experiences, but they are built from many smaller pieces often called game elements \cite{werbach2012win}. They represent parts of games used as a building blocks for creating gamified applications, as well as tools and rules that define the overall context of game \cite{gamDesElem}. This means that the definition given by the authors makes clear distinction between gamification and other systems that employ full-fledged games rather than elements of game design only  \footnote{Data source \cite{deterding2011game}}. 
\begin{figure}[h]
    \centering
    \includegraphics[width=0.75\textwidth]{gamification-btw-game-and-play}
    \caption{The matrix distinguishing the concepts related to gamification \cite{deterding2011game}.}
    \label{fig:mesh1}
\end{figure}\\\\
Furthermore, it does not include all game elements either. Based on the definition, gamification includes only a subcategory of elements called game design elements that are used as seen appropriate in the current situation. The final aspect of the definition is that gamification operates in ``\textit{non-game context}''. A non-game context refers to applications which main purpose is beyond pure entertainment. That is, using game design elements in a context ``\textit{other than games}''. This implies that gamification can be used and successfully applied to almost anything: from business, finance, personal improvement to education, health, and fitness \cite{deterding2011game}. Thus, the challenge of gamification is to select elements that normally operate within the game universe and apply them effectively in the real world.
The concept of gamification is closely related to similar pre-existing concepts such as serious games, playful design, and toys. The proposed definition aims at separating the concept of gamification from similar phenomena on a two-by-two matrix. In Figure \ref{fig:mesh1}, along one axis a distinction between gaming and playing is made, and on the other between whole game and an artifact with game elements. Gameful design or gamification differs from playful design because the former focuses on activities that are goal oriented and structured by rules while the latter focuses on activities that are based on improvisation and are free of form. Moreover, gamification is situated in the quadrant involving games and game elements, meaning that gamification makes use of gameful design rather than playful design and game elements rather than full-fledged games. This is different to serious games used also in non-game contexts, a group that includes full games that have been created for reasons other than pure entertainment. One thing to point out is that even though gamification utilizes game principles and design elements, it is envisioned  and developed as a process which sole purpose is far removed from the objectives' of traditional game design \cite{seaborn2015gamification}. Hence, the process of gamification design is partly different
from game design since the former is being used to enhance engagement in various ``\textit{non-game contexts}'' and is directed towards achieving a particular goal, whereas the latter starts from the desire to make something that people will enjoy and is completely directed towards pure entertainment \cite{seaborn2015gamification, gamifDesign}. 
\subsection{A Primer on Exergames}
Recent progressions in ubiquitous technologies offer a solution that could dispute a number of potential barriers preventing individuals to engage in regular physical activities. This solution comes in a form of video games that are developed for a certain purpose other than entertainment alone, mainly for the context of health and fitness, named exergames. They represent enjoyable tools that can increase the energy expenditure during game play, motivate players to engage in physical activity more regularly, promote social interaction, and even enhance cognitive performance \cite{staiano2011exergames}. Compared to the term gamification, the term exergame or \textit{exergaming} has been known for a while, and its roots can be found in games released in the late eighties. \\\\The name exergame is a concatenation of the words \textit{exercise} and \textit{game}, sometimes referred to as \acrfull{avg}  \cite{altamimi2012survey}. This genre includes video games with the aim of encouraging and facilitating physical activity which rely on technology that tracks body movement  \cite{altamimi2012survey}. A large amount of research has been put into exergames development, and there exist several successful commercial
products today. The \textit{Nintendo Wii} \cite{wii}, released in November 2006 for the home entertainment market was the first mainstream game console which contained a built in exergaming system. Nintendo Wii exergame contributed to a 73\% increase in Nintendo's net sales, with 24.5 million consoles and 148.4 million software units sold to date, making it the second highest selling video game in 2007 \cite{staiano2011exergames}. Apart from exercise, exergames have been used in fields such as art and education \cite{altamimi2012survey}. Researches found the the usage of exergames in these fields has led to the 
development of educational and social skills \cite{altamimi2012survey}. Playing exergames can increase caloric expenditure, heart rate, and coordination. While psychosocial and cognitive impacts of exergames may include increased self-esteem, social interaction, motivation, attention, and visual–spatial skills \cite{staiano2011exergames}. Over the recent years, the usage of exergames has also been studied for their potential health and rehabilitation benefits. This includes a diverse patients population, such as individuals with multiple sclerosis, Parkinson’s disease, stroke, and
obesity \cite{taylor2015use, barry2014role, webster2014systematic}.
Taking into account all the mentioned benefits one can gain with exergames, it is understandable that there is also an increase of writing on this topic (Figure \ref{fig:pubEx}). As in Figure \ref{fig:pub}, this one includes only the number of publications for every year for the terms ``exergame'' or ``exergaming'', and excludes patents and citations \footnote{Data source: www.scholar.google.com}. \\
\begin{figure}[h]
    \centering
    \includegraphics[width=0.9\textwidth]{pubEx}
    \caption{Search hits on term ``exergame'' or ``exergaming''.}
    \label{fig:pubEx}
\end{figure}\\\\
Exergames do an excellent job of implementing various gamification techniques that play off person's desire to master certain skills or achieve a specific goal. By breaking down the barriers to traditional exercise and workouts, they have the potential to promote physical activity and stimulate behavioral change, within a fun, enjoyable and motivating context. Researchers agree how incorporating exergames into schools, fitness centers, and homes can promote healthy youth development and even combat the childhood obesity crisis \cite{staiano2011exergames}.
\paragraph{Defining Exergames}
The goals that are set in exergames can only be achieved through body movements that are performed by the user. Hence, it is understandable that, as \cite{matallaoui2017effective} point out, ``\textit{one of the most prominent fields where
gamification and other gameful approaches have been
implemented is the health and exercise field}''. Even though known for decades, due to the technological advancements which allow more widespread and affordable usage of motion based controllers, these gameful systems and approaches that involve physical activity as the means of interacting with the game, commonly known as \textit{exergames}, have only been proliferating 
in recent years \cite{matallaoui2017effective}. 
According to \cite{kiili2010developing}, the main reason for increased interests in exergames is the concern over high levels of obesity in Western society. Apart from high calorie diet, physical inactivity is considered to be the main reason for obesity, especially among children. Since playing video games is a common leisure time activity among people of all ages, it has been argued by researchers \cite{kiili2010developing} that video games are one of the main reasons for the decreased level of everyday physical activity and hence, increased level of obesity \cite{vandewater2004linking}. This is what the emerging exergames genre tries to change by encouraging players to perform physical movements during gameplay \cite{kiili2010developing}. Exergames can be defined as ``\textit{video games that require physical activity in order to play}'' \cite{oh2010defining}. However, a more precise definition of exergame is introduced in \cite{oh2010defining}:
\begin{quotation}
\textit{``An exergame is a video
game that promotes (either via using or requiring) players’ physical movements (exertion) that is
generally more than sedentary and includes strength, balance, and flexibility activities.''}.
\end{quotation}
The authors \cite{oh2010defining} also define exergaming as an: 
\begin{quotation}
\textit{``experiential activity where playing exergames, videogames, or computer-based is used to promote physical activity that is more than sedentary activites and also includes strength, balance,
and flexibility activities''}.
\end{quotation}
The main goal of exergames is to motivate people to exercise by providing a ``\textit{safe, entertaining and engaging fitness atmosphere}'' \cite{altamimi2012survey}. Thus, one of the challenges of exergame is to make a game appealing to players, and at the same time, make it effective and adequate as an exercise. \\
%TODO: Wearable devices? THIS CAN go later in benefits: %https://scholarspace.manoa.hawaii.edu/bitstream/10125/41560/1/paper0411.pdf
%Researcher at ... showed the exergames can also be used as a fitness tool to help improve and increase the fitness level The author in [55] demonstrated this idea using his own body as the main subject of his experiment. He played three active video games which differed in the interactions they involved. The three active games that were used were Dance Dance Revolution which requires the full body interaction, the \textit{EyeToy} games which mainly require upper body interaction and \textit{GameBike} games which utilize lower body interaction. After three months of daily 30 minute sessions plays, two benefits were observed by the author; weight loss and blood sugar level reduction\\
The player forms the root of gamification as well as exergames and, in any system, the outcome is affected and driven by his motivation \cite{zichermann2011gamification}. Therefore, to understand the potentials and fundamental aspects behind gamification and exergame, one important part is to understand what drives people's motivation. Thus, in  order  to  create  an  effective  gamified system, one needs to understand  how  human  nature  works  and  how  it can be influenced and shaped. For this reason, the next sections introduce different views from psychology about motivation, explain what has to be considered in terms of truly engaging individuals and how gamification can use this in order to achieve its purpose. 

\section{Theories of Motivation}
\label{chapter:motivation}
In this section we provide a suitable overview of the subject itself and introduce terms that will be used later in the discussion. We also present theories that describe and explain various psychological effects that games have on players and how they can be used to enhance user's engagement and motivation when interacting with a gamified system. Two important theories that are regarded as crucial foundations for the concept of \textit{gamification} are presented. First, the \acrfull{sdt} by Richard M. Ryan and Edward L. Deci is introduced \cite{deci1994promoting, ryan2000intrinsic, ryan2000self, deci2000and}. Following, \acrfull{flow} by Mihaly Csikszentmihalyi is reviewed and discussed \cite{csikszentmihalyi1996flow}. 
\subsection{The Rules of Motivation}
The main purpose of gamification is to ``\textit{help people get from point A to point B in their lives}'', whether it is visiting the gamified system more often, learning a new language, or exercising more \cite{gamificationPurpose}. Gamification is about stimulating individuals to act in a certain way, or at least to develop an inclination for a certain behavior. The root of gamification is human motivation.\\The word \textit{motivation} originates from Latin \textit{motivus} and stands for ``\textit{serve to move}''. In other words, motivation can be interpreted as ``\textit{to be moved to do something}''\cite{ryan2000intrinsic}. It can be defined as ``\textit{those forces within an individual that push or propel him to satisfy basic needs or wants}'' \cite{pardee1990motivation}. One of the most influential researchers in the domain of human motivation and behavior, Richard M. Ryan and Edward L. Deci, argue that people ``\textit{can be moved}'' to act by various types of factors, as so with highly diverse experiences and consequences \cite{ryan2000intrinsic}. For example, people can be motivated because they value the activity they perform, or because there exists some external influence and pressure. Furthermore, they point out that each person has different amounts and also different types of motivation. That is, each person is different in \textit{level} (i.e. amount) and \textit{orientation} (i.e. type) of their motivation, whereas orientation might be a goal which gives rise to action and therefore governs human behavior. Gamification taps exactly in these forces within individuals that push or propel them to satisfy certain needs or wants. \\It exposes complex, but learnable, systems that individuals can engage with to achieve personal mastery and, hence, meet their objective \cite{gamificationPurpose}. There exist various motivation theories that address different aspects of motivational properties of gamified systems. However, only two theories are further discussed in detail because they have already been applied to great number of digital systems which makes them a good starting point in understanding well gamification and its influence on players' motivation.

\subsection{Self Determination Theory}
One of the most influential motivational theories is the \acrlong{sdt} introduced by Ryan and Deci \cite{deci1994promoting, ryan2000intrinsic, ryan2000self, deci2000and}. It is an empirically derived theory of human motivation that makes distinctions between different types of motivation in terms of reasons and goals that cause the respective action. That is, SDT argues that intentional human behaviors might vary in the extent to which they are \textit{self-determined} versus \textit{controlled}. This means that behaviors can vary in the extent they are experienced as being freely chosen and coming from one's self, in contrary to being pressured or controlled externally. When these behaviors are experienced as freely chosen they are considered self-determined or autonomous whereas the extent they are experienced as coerced, they are considered controlled \cite{deci1994promoting}. Having this in mind, SDT distinguishes between \textit{intrinsic} and \textit{extrinsic} motivation \cite{ryan2000intrinsic}. The first type of motivation, as the word \textit{intrinsic} already suggests, refers to performing an activity for the inherent satisfaction. When intrinsically motivated, a person is moved to act because the activity is challenging, interesting and enjoyable on its own rather than because of some external prods, pressures, or rewards. On the other hand, extrinsic motivation refers to performing an action because it leads to ``\textit{separable outcome}'' \cite{ryan2000self}. That is, there is some external reward or influence which drives the person to accomplish the task. The comparison between people intrinsically and those extrinsically motivated reveals that the former have more interest, excitement, and confidence which in turn, can not only enhance performance, persistence, and creativity but consequently boost vitality, increase self-esteem, and general well-being \cite{ryan2000self}. Though this division is for most people intuitively understandable, it is not always as clear as it may seem. For example, as the SDT theory states, ``\textit{motivations are fluid}''. Hence, people can convert extrinsic motivators to intrinsic if they internalize the desire to do so. To put it differently, in a situation where the extrinsic motivator is found meaningful, pleasurable, and consistent with a person's worldview, it can be perceived and adopted as if it was intrinsic \cite{zichermann2012}.
Although, in one sense, intrinsic motivation can exist within an individual, in another sense, it can exist in the relation between the individual and the activity one performs. Having that in mind, it is important to point out that not everyone is intrinsically motivated for the same activities and that not everyone is intrinsically motivated for any particular activity \cite{ryan2000intrinsic}. \\In SDT, the \textit{basic psychological need satisfaction} is assumed to be the core motivational mechanism that directs human's behavior. \\SDT postulates three innate psychological needs (Figure \ref{fig:ss}), that are ``\textit{essential for ongoing psychological growth, integrity, and well-being}'' and all three of them play a necessary part in optimal development, hence none can be disregarded without significant negative consequences \cite{deci2000and}. These needs are the need for \textit{autonomy}, \textit{competence} and \textit{relatedness}. When individuals experience them, they become self-determined and intrinsically motivated to pursue things that interest them the most \cite{deci2000and}.
\begin{figure}[h]
    \centering
    \includegraphics[width=0.5\textwidth]{ss}
    \caption{Basic psychological needs according to Ryan, R.M. and Deci E.L. \cite{deci1994promoting} }
    \label{fig:ss}
\end{figure}\\
%https://selfdeterminationtheory.org/SDT/documents/2010_VandenBroeckVansteenkisteNSscale_JOOP.pdf
The basic psychological needs according to Ryan and Deci are as follows \cite{deci1994promoting}:
\begin{itemize}
\item \textbf{Autonomy} represents individuals' innate desire to feel ``\textit{free}'', to experience a sense of choice and psychological freedom when carrying out certain activities \cite{deci2000and}. Situations in which individuals are provided with the opportunity to choose freely, accompanied with a positive feedback, have been shown to influence and improve autonomy and, hence, the intrinsic motivation of individuals \cite{ryan2000self}. For example, students are  autonomous when they willingly spend time and energy for completing their assignments. 
\item \textbf{Competence} represents individuals' innate desire to feel ``\textit{effective}'' when interacting with the environment. For example, students are competent in cases when they feel they can meet the challenges of their schoolwork. Furthermore, Ryan and Deci point out that positive feedback can signify effectance and provide a satisfaction of the need for competence and consequently enhance intrinsic motivation. Contrarily, negative feedback that convey ineffectance, tend to diminish the sense of competence and hence undermine individuals' intrinsic motivation. 
%A. P. Hill, "A Brief Guide to Self-Determination Theory," September 2011. [Online]. Available: http://www.heacademy.ac.uk/assets/hlst/documents/projects/round_11/r11_hill_guide.pdf. [Accessed 10 April 2013].
\item \textbf{Relatedness} corresponds to experiencing meaningful ``\textit{connection}'' to others. To put it differently, relatedness corresponds to ones innate need to to be a member of a group, to love and care, and to be loved and cared for \cite{broeck2010capturing}. This psychological need is satisfied when individuals experience a sense of togetherness and develop a close relationship with others.
\end{itemize}
The specification of autonomy, competence, and relatedness is important because it allows the prediction of variables that can affect individuals' intrinsic motivation and the development of their extrinsic motivation \cite{deci1994promoting}. Gamification and exergames achieve these needs by means of diverse game elements, which will be discussed in detail in the subsequent sections.\\
Despite the observable evidence that humans, in general, can have intrinsic motivational tendencies towards some activities, this bias appears to manifest only in certain conditions and circumstances. Hence, \acrshort{sdt} also places much emphasis on understanding conditions that enhance and sustain versus subdue and diminish intrinsic motivation \cite{ryan2000intrinsic}. A sub-theory of \acrshort{sdt} called \acrfull{cet} focuses on social and environmental factors that promote or undermine this type of motivation. It uses language that reflects the assumption that intrinsic motivation is rather catalyzed than caused when individuals are in appropriate socio-enviromental circumstances \cite{ryan2000intrinsic, ryan2000self}. In other words, intrinsic motivation does not occur by itself, but represents the outcome of one's interaction with the environment and one's interests and preferences. That is, intrinsic motivation ``\textit{will flourish if circumstances permit}'' \cite{ryan2000self}. Furthermore, \acrshort{cet}, which focuses mainly on the fundamental needs for competence and autonomy, argues that interpersonal events and structures, such as rewards, communication or feedback can increase intrinsic motivation for certain actions because they satisfy the basic psychological need for competence. Accordingly, it is predicted that optimal challenges, positive feedback and freedom from degrading evaluations promote intrinsic motivation, while tangible rewards, threats, deadlines  and  directives decrease it \cite{ryan2000self}. \acrshort{cet} also argues that the satisfaction of the psychological need for competence will not enhance intrinsic motivation unless it is joined by a sense of autonomy. Hence, people must perceive that their behavior is self-determined in order for intrinsic motivation to be maintained or enhanced. In other words, for a high level of intrinsic motivation, the needs for competence and autonomy must both be satisfied \cite{ryan2000self}. It is important to point out, as stated by Ryan and Deci, that people will be intrinsically motivated for certain activities only when they are intrinsically captivating for an individual. This includes activities that offer a degree of novelty, challenge or aesthetic value. Activities that do not provide such appeal, will not be experienced as intrinsically motivated. \\
Even though intrinsic motivation is of great importance, most of the activities people engage in are not intrinsically motivated. Such activities require an \textit{external push} in order to be realized. This motivation, contrary to intrinsic motivation which refers to doing an activity simply for the enjoyment of the activity itself, is known as \textit{extrinsic} motivation. It refers to performing certain activities because it is expected to result in some additional outcome or reward that have an instrumental value for the individual performing that action \cite{ryan2000self}. In general, extrinsically motivated behaviors are those which would not happen instinctively, and hence must be prompted by an intrumentality \cite{deci1994promoting}. Various studies demonstrated that in specific circumstances extrinsic motivation can sustain intrinsic motivation, thus suggesting that extrinsically motivated behaviors can also be self-determined \cite{deci1994promoting}. Extrinsically motivated behaviors become self-determined through the process of \textit{internalization} and \textit{integration}.\\ Internalization involves transforming external regulatory processes into internal regulatory processes, while integration corresponds to the process of integrating these newly internalized values and regulations into one's self \cite{deci1994promoting}. There exist four types of extrinsic regulation that can result from different types of internalization and integration, which were introduced within \acrshort{sdt} as a subtheory called \acrfull{oit} \cite{deci1994promoting, ryan2000intrinsic, ryan2000self}. For instance, students who work on their assignments because they personally understand its importance for their future career and those who do it only to adhere to their parents' control are both extrinsically motivated. Even though both cases involve instrumentalities rather than enjoyment, the former entails personal endorsement and a feeling of choice while the latter is associated only with an external regulation.\\
Figure \ref{fig:tax} illustrates the \acrshort{oit} taxonomy of motivational types arranged from left to right in terms of the degree to which the motivation originates from the self (i.e. are self-determined).\\ 
\begin{figure}[h]
    \centering
    \includegraphics[width=\textwidth]{sdt}
    \caption{Types of motivations based on Self-Determination  Theory by Ryan, R.M. and Deci E.L. Adapted from  \cite{ryan2000intrinsic}}
    \label{fig:tax}
\end{figure}\\
First, the extrinsically motivated behavior that is the least autonomous is known as \textit{external regulation} and is regulated through some external means, such as rewards and constraints. For example, an athlete who participates in the Olympics only to obtain a medal represents an instance of externally regulated behavior. In case of \textit{introjected regulation}, individuals begin to internalize the reasons for their action. However, this internalization only replaces the external source of motivation with an internal one, such as guilt, worry or shame. That is, when people are motivated to perform activity in order to maintain feeling of worth. An example for introjection is an athlete who goes to the practice just because she would feel guilty if it has been skipped. A more autonomous type of extrinsic motivation, \textit{identification}, manifests when a person identifies with the importance of some behavior and accepts it as a personal regulation only because it benefits the person in achieving a specific goal. \\An example for this behavior is an athlete who does not like weight lifting, but nevertheless chooses to to do it because it will positively impact her future performance. \textit{Integrated regulation}, the most autonomous of extrinsic motivation that shares many qualities with intrinsic motivation, is a form of motivation that arises when an individual has fully assimilated the identified regulation within herself. An example of integrated regulation is an athlete who chooses to postpone the night out with friend in order to be in good shape for the next day's tournament. Integration together with intrinsic motivation represent the core for self-determined functioning and they both share the qualities that constitute self-determination. Even though they might seem quite similar, they are different in the sense that intrinsically motivated behaviors are ``\textit{autotelic in nature}'' while, on the other hand, integrated behaviors are ``\textit{instrumentally (though freely) performed}'' for the outcome that is self satisfactory.  Finally, the self-determination continuum is closed with  \textit{amotivation} which represents ``\textit{non-regulation}'' from the SDT perspective as it refers to a state where intentions to act are non existent. A person amotivated towards exercise would not exercise at all, 
or engage in exercise in a passive and disorganised  manner \cite{deci1994promoting, ryan2000intrinsic, vallerand2007intrinsic}.\\\\
Having outlined the basic concepts behind \acrshort{sdt}, the next section covers the theory of \textit{flow}.
\subsection{State of Flow}
%TODO: is the state of flow realistically achievable in 15 minutes or short intervals? to argue. so you need to look into how much time is needed to reach this state?\\
Another approach for describing the foundations of motivation is \acrlong{flow}. Mih\'{a}ly Cs\'{i}kszentmih\'{a}lyi, one of the most recognized game psychologists and a professor at University of Chicago, described in 1975 for the first time the phenomenon of \textit{flow}. Being fascinated by artists who would essentially get lost in their work Cs\'{i}kszentmih\'{a}lyi argued how, creative people might differ from one another in many ways but they always have one thing in common. They love what they do. Their love for a particular activity is not because of a potential outcome or a reward. What drives them is solely the opportunity to do what they enjoy doing \cite{csikszentmihalyi1996flow}. Athletes often refer to this concept as ``\textit{being in the zone}'', religious mystics
as ``\textit{being in ecstasy}'', artists and musicians as ``\textit{aesthetic rapture}'' \cite{csikszentmihalyi1997finding}. After a series of studies, based on the individuals' responses regarding their emotions while performing certain activity they enjoy, Cs\'{i}kszentmih\'{a}lyi  developed a theory of optimal experience based on the concept of \textit{flow}, which he describes as the
\begin{quotation}
``\textit{the state in which people are so involved in an activity that nothing else seems to matter; the experience itself is so enjoyable that people will do it even at greater cost, for the sheer sake of doing it}'' \cite{flow1990psychology}.
\end{quotation} 
Flow is also considered as an optimal state of intrinsic motivation, where people become absolutely immersed in what they are doing, they forget about physical feelings, passage of time, and their ego fades away \cite{lithiumGamification}. 
It represents a state in which one feels in control, fully immersed and motivated, at the top of its abilities and neither overwhelmed by difficulty nor uninterested. Cs\'{i}kszentmih\'{a}lyi states that flow experiences are relatively rare in everyday life, however, various activities are able to produce them, provided certain conditions are met \cite{csikszentmihalyi2014flow}. The activities inducing the state of flow do not have to be of complex nature. That is, the flow can occur during most complex surgical operation or during a simple card game. Kowal and Fortier (1999) have also pointed out that flow can occur in a myriad of life domains, such as work, sports and physical activity, school, and leisure \cite{kowal1999motivational}. Cs\'{i}kszentmih\'{a}lyi further argues that three conditions have to be met in order to achieve a flow state. First, a state of flow needs clearly defined set of goals which must guide the person, and give purpose to the behavior (\textbf{clear goals}) \cite{csikszentmihalyi2014flow}. The second condition for obtaining the state of flow is the presence of clear and immediate feedback (\textbf{unambiguous
feedback}). It informs the person if a specific goal is met and how to adjust performance according to the ``\textit{continually  changing environment demands}'' \cite{csikszentmihalyi2014flow}. Lastly, one of the most important condition is to maintain balance between perceived challenges and perceived skills (\textbf{challenge-skill balance}) \cite{csikszentmihalyi2014flow}. When experiencing flow, both the challenge and the skill set, required to meet the challenge, need to be balanced and at an individually high level. Nakamura and Cs\'{i}kszentmih\'{a}lyi further argue that under these three conditions, individuals can enter a state with the following characteristics \cite{nakamura2014concept, csikszentmihalyi2014flow}:
\begin{itemize}
\item \textbf{Control}. A sense that one has skills sufficient enough to minimize the possibility of any mistake or error. Hence, one can fully enjoy the current situation because nothing can emerge as a surprise
\cite{csikszentmihalyi2014flow}. This sense of control is believed to be one of the important flow antecedents in games \cite{kiili2006evaluations}. 
\item \textbf{Action–awareness merging}. This implies that the flow state is so involving that it affects the individual in a way that the activity performed becomes spontaneous, automatic and natural.
\item \textbf{Concentration}. While in flow, one experiences intense and focused concentration on what is being done in the present moment. By doing so, one is able to forget all unpleasant things beyond the performed activity since the person is left with no cognitive resources for irrelevant information processing \cite{kiili2006evaluations}. 
\item \textbf{Loss of self-consciousness}. During the flow, the \textit{self} disappears from one's awareness. That is, while thoroughly engrossed with an activity, as in the state of control, few cognitive resources are
available for self-scrutiny \cite{kiili2006evaluations}.
\item \textbf{Distortion of temporal experience}. Typically, the sense of time during the flow experience tends to bear little relation to the actual passage of time. To put it differently, in a state of flow, one feels that time passes faster than normally.
\item \textbf{Autotelic experience}. Often, this refers to an activity that is performed simply because it is intrinsically rewarding and not with the expectation of some future benefit. It is also referred to as the end result of other conditions and characteristics that induce flow. 
\end{itemize}
Whenever individuals try to reflect on their flow experiences, they tend to mention some and often all of these characteristics. The described conditions and characteristics of flow are known as the ``\textit{nine dimensions of the state of flow}'', where the first five dimensions can be considered as ``\textit{flow antecedents}'' and the rest indicators of ``\textit{flow experience}'' \cite{kiili2006evaluations}. According to Cs\'{i}kszentmih\'{a}lyi, flow often tends to occur in situations when we face challenges that match our skills and abilities. That is, it occurs when we perform tasks and activities that are neither too difficult nor too easy with respect to the set of skills we possess, a balance of the relationship between challenge and ability \cite{csikszentmihalyi1996flow, csikszentmihalyi1997finding, flow1990psychology}. This balance is referred to as \textit{flow zone}. When the task is too difficult (i.e., the skill cannot meet the challenge), that is when one is above the flow channel, we are likely to experience anxiety. In the opposite case, when the task is slightly too easy and task challenges do not come close to our ability, the result is boredom.
\begin{figure}[h]
    \centering
    \includegraphics[width=0.7\textwidth]{flowZone}
    \caption{The flow channel \cite{csikszentmihalyi1996flow}}
    \label{fig:flowZone}
\end{figure}\\
Figure \ref{fig:flowZone} depicts the graphical representation of the state of flow, where y-axis represents the difficulty of the challenge and the x-axis skill set required to meet the specific challenge. The diagram also contains the flow-channel, as well as the anxiety-region and the boredom region.\\
Over the years, new theories regarding the state of flow have been introduced and the concept flow was redefined by introducing eight experimental channels rather than previously mentioned quadrants \cite{nakamura2014concept}. Figure \ref{fig:flowModel} shows the refined \textit{challenge} - \textit{skill} space which now contains a series of concentric rings, associated with increasing intensity of experience. Based on the current model of the flow state (Figure \ref{fig:flowModel}), the flow is experienced in situation when challenges and skills are above the individual's average levels. When the task is slightly too easy (or slightly too hard) we fall out of the state of flow and enter a state where we feel in control (or the state where we feel aroused if the task is slightly too hard). When the difficulty of the task performed is above our skills, we are tend to experience anxiety. On the other hand, if challenges do not come close to our ability, we tend to experience boredom. \\
\begin{figure}[h]
    \centering
    \includegraphics[width=0.7\textwidth]{flow-model}
    \caption{The current model of the flow state \cite{nakamura2014concept}}
    \label{fig:flowModel}
\end{figure}\\
In cases when the challanges and our skills are at relatively low level, apathy is experienced. The new model also deals with the intensity of the experience. Presented by the concentric rings in Figure \ref{fig:flowModel}, it can be noticed that the experience level increases with distance from person's average levels of challenge and skill \cite{nakamura2014concept}. Cs\'{i}kszentmih\'{a}lyi also argues how sports and games are more likely to lead to a flow state since they usually have clear goals and feedback structures. However, a given individual can find flow in almost every activity that for some other individuals might seem boring or tiresome \cite{csikszentmihalyi2014flow}.
\subsubsection{Flow, Gamification, and Exergames}
In the context related to human behavior and computers, the concept of flow has been mostly studied in video games, human-computer interaction, and instant messaging, to name a few \cite{hamari2014measuring}. Currently, there exist only few studies investigating flow in the context of gamification \cite{hamari2014measuring, sillaots2014achieving}. Thus, there is insufficient data to draw conclusions as to which of the nine dimensions of flow discussed previously would be most important in the context of gamification. To this end, a study was conducted in which the influence and importance of the different dimensions of flow in gamification is investigated \cite{hamari2014measuring}. The data for this study was gathered from users of an exercise gamification service (n = 200). As a measurement instrument for flow, the researcher have been utilized the \acrfull{dsf} model, designed to access flow experiences in physical activity \cite{jackson2002assessing}. \\\\The results showed that autotelic experience, clear goals, (immediate) feedback, control, and challenge-skill balance were the most salient dimensions of flow in gamification (of exercise). On the other hand, time transformation, merging action-awareness, loss of self-consciousness were the least salient as shown in Figure \ref{fig:dfs2}.\\
\begin{figure}[h]
    \centering
    \includegraphics[width=\textwidth]{dfs2}
    \caption{Measuring flow in Gamification: Dispositional Flow Scale-2 \cite{dfs2}.}
    \label{fig:dfs2}
\end{figure}\\
Furthermore, results also suggest that in gamifed context, the autotelic experience is highly correlated with the conditions. It has also been suggested that in the gamification context, autotelic experience truly represents a condition for reaching flow, thus implying that one can more easily reach flow if the activity is initially intrinsically motivating \cite{hamari2014measuring}.\\\\
%%related to challenging game:https://sci-hub.la/https://dl.acm.org/citation.cfm?id=1930509
%\subsection{Motivation and Sports}
%pelletier1995toward
%Vallerand (2004) states that motivation in sports matters, as it ``represents one of the most important variables in sport''. It is known to be a key element of success in sport and athletes' persistence with an exercise regiment \cite{vallerand2007intrinsic}. Intrinsic and extrinsic motivation have been particularly popular topics that allowed researchers to explain various phenomena of importance in sport and physical activity. Various studies in the domains of health, physical education, exercise and sport have explored the SDT derived hypothesis that intrinsically relative to extrinsically motivated behavior  
%TODO!!!!!!!!!!!!!!!!!!!: in which areas gamification in sports works already? are there theory parts results that are applicable to our work?...\\*\\*
Armed with a clearer understanding of the theory behind human motivation, the next chapter provides an overview of the key components of gamification. We place our focus on different player types and how gamification components can influence player's engagement, motivation, and game enjoyment. %The main goal of the chapter is to provide an overview of the most relevant key elements of Gamification relevant for the development of iMMotion, in particular why certain game mechanics were the best choice for the iMMotion project.
\chapter{Gamification Elements}
An important aspect of game thinking is that players differ from one another and their motivation for engaging in gaming activities should not be generalized. That is, people choose to play games for different reasons, and thus, the same video game can have a different meanings or consequences for different players \cite{yee2006motivations}. The more is known about who is playing the game, the easier it is to design and implement an experience that will drive players' behavior in the desired way \cite{zichermann2011gamification}. Hence, knowing more about player types and what drives them forward, can help us to design a gamified system that will be utilized more likely by our target audience. The subsequent sections will further explore premises about player behavior and corresponding personality types.
\section{Bartle's Four Player Types}
One way to understand players' motivation is to leverage the work accomplished by Richard Bartle in examining player types. Bartle conducted researches in the area of game design and development, and analyzed the ethnography of online game players in the first \acrfull{mud} in 1978 \cite{mud}. In order to understand why people play games, he identified four main player personality types of \acrshort{mud} according to specific psychological aspects of their personality and how they prefer playing in a virtual world: \textit{Explorers}, \textit{Socializers}, \textit{Killers}, and \textit{Achievers} \cite{bartle1996hearts}. The player personality types, as depicted in Figure \ref{fig:userTypes}, can be defined as follows:
\begin{itemize}
\item \textbf{Explorers} represent players which are driven by motivation to ``\textit{find out as much as they can about the 
virtual  world}'' \cite{bartle1996hearts}. Not only they enjoy exploring every corner of the game environment and searching for interesting features (i.e. bugs), but also understanding  how 
everything functions \cite{bartle1996hearts}. 
In a sense, for this type of players ``\textit{the experience is the objective}'' \cite{zichermann2011gamification}.
\item \textbf{Socializers} are player who play games for the benefit of a social interaction \cite{zichermann2011gamification}. They usually enjoy using communication tools that are provided by the game in order to engage in conversation with other players. 
\item \textbf{Achievers} are goal (achievement) oriented players. They are players who are proud of their ``\textit{formal status in the game's built-in level hierarchy}'' and also ``\textit{of how short a time they took to reach it}'' \cite{bartle1996hearts}. This type of players are competitive who enjoy beating difficult challenges whether they are explicitly  set by the game (e.g.  leveling  up  or  gathering  points) or by themselves (e.g. accumulating as much virtual money as possible). 
\item \textbf{Killers}, also known as \textit{griefers} \cite{zichermann2011gamification}, are the smallest population of all the player types. They are very similar to achievers in their motivation for winning, however, these players obtain  enjoyment from causing anxiety and "\textit{imposing  themselves  on  others}" \cite{bartle1996hearts}. For them, winning is only meaningful if someone else loses.
\end{itemize}
\begin{figure}[h]
    \centering
    \includegraphics[width=0.9\textwidth]{userTypes}
    \caption{Bartle's taxonomy of player types \cite{bartle}.}
    \label{fig:userTypes}
\end{figure}
In Figure \ref{fig:userTypes}, axes represent the source of players' interest in  \acrshort{mud}. The horizontal axis depicts the player's preference for interacting with other players vs. interacting with the world. The vertical axis represents the player's preference for (inter)acting with something vs. (inter)acting on something. Thus, according to Figure \ref{fig:userTypes}, achievers prefer to act on the world, while socializers prefer to interact with other players \cite{bartle}. It is important to point out that people are not exclusively one or another of the presented player types \cite{zichermann2011gamification}. Zichermann and Hunter \cite{zichermann2011gamification} argue that most people have some percentage of each type and the most dominant type will probably change throughout the individual's life.\\ Even though Bartle's player types have not been designed in particular for gamification it can help in understanding what attitudes may be dealt with when implementing a gamified solution. 
\section{Marczewski's User Types Hexad}
Bartle's taxonomy of user types was created
specifically for \acrshort{mud} and it should not be generalized to other game genres nor to gameful design \cite{tondello2016gamification}. Moreover, it does not consider players who are extrinsically motivated. In order to address these issues Marczewski proposed six user types that differ in the degree to which they can be motivated by either intrinsic or extrinsic motivational factors \cite{tondello2016gamification} and introduced the \acrlong{hexad} that is depicted in Figure \ref{fig:playerHex}.
\begin{figure}[h]
    \centering
    \includegraphics[width=0.75\textwidth]{playerHex}
    \caption{Gamification user types hexad \cite{tondello2016gamification}.}
    \label{fig:playerHex}
\end{figure}\\
\acrshort{hexad} is developed in order to identify the users of the gamified system more efficiently. It is based on users' intrinsic and extrinsic motivations as defined by \acrshort{sdt} and enables accurate measures of user preferences \cite{tondello2016gamification}. Marczewski identifies the following player types:
\begin{itemize}
\item \textbf{Socialisers}. Individuals who are motivated by \textit{Relatedness}. They prefer to interact with others players and create social connections with them \cite{tondello2016gamification}.
\item \textbf{Free Spirits}. Individuals who are  motivated by \textit{Autonomy} and self-expression. They enjoy the freedom of expression and acting without any external control \cite{tondello2016gamification}.   
\item \textbf{Achievers}. Individuals who are  motivated by \textit{Competence}. As in Bartle's taxonomy, this group seek progress withing the gamified environment by completing various tasks and enjoy proving themselves by overcoming difficult challenges \cite{tondello2016gamification}.
\item \textbf{Philanthropists}. Individuals who are motivated by \textit{Purpose} and \textit{Meaning}. These individuals enjoy giving to others with no expectation of reward in return \cite{tondello2016gamification}.
\item \textbf{Players}. Individuals who are motivated by \textit{Extrinsic rewards}. This player type is motivated only by the reward offered by the gamified system. They will do anything necessary to obtain the extrinsic reward independently of the type of the activity \cite{tondello2016gamification}.
\item \textbf{Disruptors}. Individuals who are  motivated by \textit{Change}. In general, they tend to disrupt the system either directly or through other users in order to force positive or negative changes \cite{tondello2016gamification}.
\end{itemize}
As already mentioned, most people demonstrate each player type to a certain degree. Understanding these player types will support 
the process of choosing the game elements that will be most appealing for the target audience and drive the 
desired behavior. Furthermore, adding features and content in order to appeal to different player types can be of great help to diversify the audience of the gamified system, and create enjoyable experiences for many players.

\subsection{Game design elements}
% \cite{schobel2016agony}.% Sch{\"o}bel \textit{et al.} carried out a literature review to analyze the gamification elements used in various research studies.
%prezentacija o game vs play http://gamification-research.org/2012/04/defining-gamification/
%detering definise elements of game design. five leveles.Next the most common
%Next, and overview and various classification frameworks of game elements which might further enhance engagement, the potential goal of gamification is presented. 
%In a few words, the gamification process can be described as the adoption of some techniques inherited from game design into different situa-tions, other than games. In this perspective, the application of the typical game elements and the exploitation of common game design patterns are used to the aim of making some activities more appealing. In this way, users are stimulated to complete tasks by the desire of getting some rewards (Werbach & Hunter, 2012). Hence, gamification is not related to solve difficult puzzles or avoid tricks but it is the finding of effective ways to drive individuals to their goals faster. Through gamification, people feel involved in the process and are called to be proactive so that they can empower their own abilities and enhance their attitudes both online, in virtual worlds, and offline, in real world situations. Currently, gamification is used by industries to enhance the outcome of their communication campaigns and to drive the attention of people to advertising and marketing messages, in order to maximize their outcome.To conclude, gamification requires a deep understanding of what we can learn from games, so that we can design enjoyable environments and raise passion for the game we are playing  Applying gamification techniques to enhance the effectiveness of video-lessons. Available from: https://www.researchgate.net/publication/283469412_Applying_gamification_techniques_to_enhance_the_effectiveness_of_video-lessons [accessed Mar 5, 2017].
In \cite{zichermann2011gamification}, Zimmerman and Cunningham argue that in order to achieve the greatest impact for players when creating a gamified experience, one should leverage aspects of game design by focusing on its core elements. However, as already pointed out, the goal of gamification is not to build a ``full-fledged game'', but to use game elements in order to provide a gamified experience and enrich the application to engage and motivate the users \cite{deterding2011game, werbach2012win}. Game design elements are the basic building blocks of gamification applications \cite{deterding2011game, werbach2012win}. With respect to the usage of game elements in gamified system, there have been various attempts to group and classify them based on certain criteria \cite{deterding2011game, werbach2012win, kapp2012gamification, zichermann2011gamification}. The most broadly used game elements that are present in most gamified systems are points, badges, and leaderboards. These elements are commonly known as \acrshort{pbl} \cite{werbach2012win}. Derived from the available literature, \cite{deterding2011game} found that game elements previously identified and presented in different research studies, fell into five distinct levels of abstraction. Table \ref{table:gameElements} presents a model for classifying game elements. It has five levels of abstraction, ordered from concrete to abstract \cite{deterding2011game}. On the other hand, \cite{kapp2012gamification} does not classify but only lists typical game elements such as goals, time, rules  conflict, competition, cooperation, feedback, levels,  storytelling, curve of interest, and aesthetics. Likewise, \cite{50GamElements} identifies and lists 52 (as of January 2018) elements that support various player types and can enhance gamification designs.\\\\
\begin{table}[!htbp]
\centering
\caption{Taxonomy of game design elements by level of abstraction. Adopted from \cite{deterding2011game}}
\label{table:gameElements}
\begin{tabular}{lll}
\hline
\textbf{Level} & \textbf{Description} & \textbf{Example} \\ \hline
\begin{tabular}[c]{@{}l@{}}Game interface\\ design patterns\end{tabular} & \begin{tabular}[c]{@{}l@{}}Common, successful interaction\\  design components and design \\ solutions for a known problem\\  in a context, including prototypical\\  implementations\end{tabular} & \begin{tabular}[c]{@{}l@{}}Badge, leaderboard, \\ level\end{tabular} \\ \hline
\begin{tabular}[c]{@{}l@{}}Game design\\ patterns and\\ mechanics\end{tabular} & \begin{tabular}[c]{@{}l@{}}Commonly reoccurring parts of \\ the design of a game that\\  concern gameplay\end{tabular} & \begin{tabular}[c]{@{}l@{}}Time constraint, \\ limited resources, turns\end{tabular} \\ \hline
\begin{tabular}[c]{@{}l@{}}Game design\\ principles and\\ heuristics\end{tabular} & \begin{tabular}[c]{@{}l@{}}Evaluative guidelines to approach a\\  design problem or analyze a given\\ design solution\end{tabular} & \begin{tabular}[c]{@{}l@{}}Enduring play,\\ clear goals, \\ variety of game styles\end{tabular} \\ \hline
Game models & \begin{tabular}[c]{@{}l@{}}Conceptual models of the components of\\  games or game experience\end{tabular} & \begin{tabular}[c]{@{}l@{}}
MDA; challenge, \\ fantasy, curiosity;\\ game design atoms; \\ Core Elements of the \\ Gaming Experience \end{tabular} \\ \hline
\begin{tabular}[c]{@{}l@{}}Game design\\ methods\end{tabular} & \begin{tabular}[c]{@{}l@{}}Game design-specific\\  practices and processes\end{tabular} & \begin{tabular}[c]{@{}l@{}}Playtesting,\\ playcentric design, \\ value conscious\\ game design\end{tabular} \\ \hline
\end{tabular}
\end{table}\\
In \cite{zichermann2011gamification}, researchers take a different approach and base their description of game elements on the \acrshort{mda} framework, which is categorized as a \textit{game model} in the framework proposed by \cite{deterding2011game}. It is one of the most frequently used frameworks of game design and stands for \acrlong{mda} \cite{hunicke2004mda}. 
Introduced by Robin Hunicke, Mark LeBlanc and Robert Zubek, the \acrshort{mda} framework formalizes games consumption by breaking them into their distinct elements: \textit{rules}, \textit{system}, and \textit{fun}. These elements translate into the following design counterparts which constitute the \acrshort{mda} framework: \textit{Mechanics}, \textit{Dynamics} and \textit{Aesthetics} \cite{hunicke2004mda}. Mechanics are the functioning components that make up the game, such as \acrshort{pbl}. They represent the specific elements of the game and control mechanism that are provided to the player within the game's context. Dynamics, on the other side, represents player's interactions with the mechanics. They specify how the player reacts to and interacts with the mechanics of the system, both individually and with other players. Lastly, the aesthetics of the system are the emotional reponses of the users who interact with the game system \cite{zichermann2011gamification}. This framework has been very influential in helping game designers in conceptualizing different aspects of games.\\ Another approach in classifying game elements is introduced by Kevin Werbach and Dan Hunter \cite{werbach2012win}. The authors base their classification of game design elements on the previously introduced \acrshort{mda} framework. They argue that ''\textit{game elements exist in a hierarchy}" and establish three categories of game elements that of relevance to gamification. \\\\These categories are: \textit{Dynamics}, \textit{Mechanics}, and \textit{Components}. The terms are similar to the ones presented in the \acrshort{mda} framework, although in \cite{werbach2012win} are used differently. The gamification elements in decreasing order of abstraction where each mechanic is tied to one or more dynamics, and each component is tied to one or more higher-level elements as depicted in Figure \ref{fig:mdc}.
\begin{figure}[h]
    \centering
    \includegraphics[width=0.75\textwidth]{mdc}
    \caption{The Pyramid of Game Elements from Werbach and Hunter \cite{werbach2012win}.}
    \label{fig:mdc}
\end{figure}\\
In the next section different gamification mechanics, dynamics and components are listed and described. In order to fulfill their objectives, all elements also need to be analyzed with respect to their impact on player's motivation and previously specified player types. This way, the most appropriate ones can be selected, taking into account the game context, without affecting player's motivation.
\paragraph{Dynamics}
At the highest level of abstraction are game dynamics that serve as the core, underlying framework for the gamification to take place. They represent the implicit structure that guides the game, sets up the rules and constraints, and defines the overall purpose, aim and goal of the game and gamified system \cite{werbach2012win, WerbachCoursera}. According to Werbach and Hunter, the most important game dynamics are \cite{werbach2012win}:
\begin{itemize}
\item \textbf{Constraints} (limitations or forced trade-offs). Every game has some constraints, because games create meaningful choices and challenging problems by limiting player's freedom of acting. Hence, the decision of what constraints get put on users represent an important dynamic that game designers needs to take into account. 
\item \textbf{Emotions}. Games can produce a myriad of
emotions. However, Werbach argues that the emotional palette of gamification is typically somewhat more limited \cite{WerbachCoursera}. The reason for this is because gamification deals with real world, non-game context, such as marketing, health, or exercise context. In cases like those, getting someone, for instance, really upset, will probable not be beneficial and valued.
\item \textbf{Narrative} (a consistent, ongoing storyline). It represents the structure that pulls together the various elements of the game. In case when in the gamified system exists no sense of narrative, there is a risk that it becomes a bunch of abstract and incoherent concepts randomly tied together. For instance, the player should be well aware of the mechanics behind the scoring system. When they are, the players can easily set their own goals which, in turn, can increase their motivation and engagement.
\item \textbf{Progression} (the player's growth and development) represent one of the most important dynamics in a gamified system which can stimulate the basic psychological needs like competence and relatedness. The third psychological need (autonomy) can also be stimulated by allowing players to decide the nature of their progress.
\item \textbf{Relationships} (social interactions generating feelings of camaraderie, status, altruism, to name a few).
\end{itemize}
\paragraph{Mechanics}
Kevin Werbach describes game mechanics as ``\textit{the processes that drive actions forward}''. He subsequently compared mechanics to ``\textit{verbs}'' which help people to play games \cite{werbach2012win}. These are the essential processes that generate player engagement. Werbach identifies ten important game mechanics:
\begin{itemize}
\item \textbf{Challenges} - puzzles or other tasks that require effort to solve.
\item \textbf{Chance} - elements of randomness.
\item \textbf{Competition} - one player or group wins, and the other loses. 
\item \textbf{Cooperation} - players must work together to achieve a shared goal.
\item \textbf{Feedback} - information about how the player is doing.
\item \textbf{Resource acquisition} - obtaining useful or collectible items.
\item \textbf{Rewards} - benefits for some action or achievement.
\item \textbf{Transactions} - trading between players, directly or through intermediaries.
\item \textbf{Turns} - sequential participation by alternating players.
\item \textbf{Win States} - objectives that makes one player or group the winner - draw and loss states are related concepts.
\end{itemize}
Even though all listed dynamics can potentially enhance users' engagement and motivation for certain activities, to prevent going beyond the scope of this thesis, the listed dynamics will not be further discussed in details. 
\paragraph{Components}
Components make up the largest fraction of game elements. They can be viewed as more-specific forms that game mechanics or dynamics can take. These elements are less abstract than the categories described previously, and lead to tools that can be used in order to begin incorporating gamification in the context of interest. There exist various game components that can be successfully used in gamified systems. However, some are more common than others, and some are more influential in shaping common examples of gamification. Werbach and Hunter \cite{werbach2012win} examined over 100 implementations of gamified systems and claim that three elements always appear. These elements are: \textit{points, badges}, and \textit{leaderboards}, commonly reffered to as the PBL Triad. They further point out how these elements are so common within gamification that ``\textit{they are often described as though they are gamification}'', even though they are not \cite{werbach2012win}.  In their comprehensive survey of peer-reviewed empirical studies on gamification, researchers \cite{hamari2014does} also found that these three elements ``\textit{were clearly the most commonly found variants}" in the large variety of elements tested. The same elements were listed and described by Zichermann, alongside \textit{levels}, \textit{challenges/quests}, \textit{onboarding}, and \textit{engagement loops} \cite{zichermann2011gamification}. The PBL triad represents a useful starting point for building gamified solution. However, relying only on them can lead to negative outcomes \cite{werbach2012win}. This is because elements on their own do not make the game \cite{werbach2012win, WerbachCoursera}. They represent great tools for communicating progress and acknowledging players effort, but neither points nor badges in any way constitute a game. This is where problems usually emanate. Heavily relying only on these elements, without understanding other aspects of game design, could suppress players' intrinsic motivation to engage with a gamified system. However, no actual empirical evidence exists to back this claim \cite{mekler2013points}. Furthermore, when using the PBL triad, the most emphasis is put on rewards. The problem with this, according to Werbach, is that ``\textit{not all rewards are fun; not all fun is rewarding}'' \cite{WerbachCoursera}. Also, not all player types are motivated by rewards. Thus, relying only on them, might demotivate other player types in further engagement with the gamified system. Werbach claims that the thing that can make game elements successful is the ``\textit{way they are all tied together}'', which often involves some higher level concepts such as dynamics. To sum up, even though PBLs have huge potential, they are not right for every project.\\ That is why, some other elements should also be considered in order to ``\textit{extract the maximum value from gamification}'' \cite{werbach2012win}. Apart from the described ones, Werbach and Zichermann also identified various gamification mechanics that can be used in gamified systems. Some of the most common are \cite{zichermann2011gamification}:
\begin{itemize}
\item Achievements - defined objectives.
\item Avatars - visual representations of a player's character.
\item Collections - sets of items or badges to accumulate.
\item Combat - a defined battle, typically short-lived.
\item Content unlocking - aspects available only when players reach objectives.
\item Gifting - opportunities to share resources with others.
\item Levels - defined steps in player progression.
\item Quests - predefined challenges with objectives and rewards.
\item Social graphs - representation of players' social network within the game.
\item Teams - defined groups of players working together for a common goal.
\item Virtual goods - game assets with perceived or real-money value.
\end{itemize}
According to Werbach the central task of gamification design is putting all these elements together. However, one should keep in mind that no gamification system will or need to include all of these elements. On the other hand, it is necessary to take into account all the possible elements in order to build an engaging gamified service \cite{werbach2012win}. \\\\
In the following chapters we detail the development process of our gamified system and further discuss the reasoning behind gamification elements that were incorporated. %Our main goal was to create an exergame that is enjoyable, easy to use and follow, and consequently, motivates athletes to engage in warm up sessions more often.
%http://sci-hub.la/http://www.sciencedirect.com/science/article/pii/S0747563215301229 - sdt and plb triad
%just for health and fitness gamification
%First  the  elements  most related   to   motivation   are   presented   and   then   the   most   popular elements   in gamification  literature  are  presented.  Although  all  game  elements  can be  used  in gamification they might not all be equally relevant. Some are more potent and useful than others. As it is also a goal of this paper to examine which elements are the most  relevant,  based  on  the  research  done  on  motivational  theory  and  gamification literature a distinction is made.
%In theory, any context, task or process can be gamified \cite{muntean2011raising}. %skini i procitaj Muntean!! 
%\subsection{Objectives}
%\subsection{Principles}
%\subsection{The impact of Gamification}
%\subsubsection{Gamification in sports - Effects and Objectives}
%The main goal of gamification is to engage the users Gamification’s main goal is to rise the engagement of users by using game-like techniques such asscoreboards and personalized fast feedback (Flatla et al, 2011) making people feel more ownership and purpose when engaging with tasks (Pavlus, 2010).\cite{burke2016gamify}
%\subsection{Gamification critiques}
%\subsubsection{Elements and Player Types}
%OVO OTKOMENTARISI KASNIJE!!!!!!!!!!!!!!!!!!!!!!!
%\begin{appendices}
%\addtocontents{toc}{\vspace{2em}}
%  \chapter{Major Causes of Sports Injuries}
%  \label{appendix:injuries}
%  \begin{figure}[h]
%    \centering
%    \includegraphics[width=0.7\textwidth]{injuries_types}
%    \caption{Injury depends on independent factors: intrinsic and extrinsic \cite{lefevre2016major}}
%    \label{fig:injuries_types}
%\end{figure}
%\end{appendices}